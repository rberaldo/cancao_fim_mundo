\chapter{Sozinhos}

\data{São Paulo, 2 de outubro de 2003}

Em seu décimo-oitavo aniversário, no dia em que se tornava um homem, Ian
Abercrombie teve de testemunhar a morte de sua mãe.

Laura Ferraz Abercrombie fora, em sua juventude, uma atraente modelo
fotográfica. Mesmo tendo nascido em berço de ouro, com direito a metade
dos bens de seu rico pai, ela escolheu seguir uma carreira para se
sustentar. E foi durante um de seus desfiles de roupa de banho que
conheceu um jovem escocês que estava de passagem pelo Brasil. Ambos
conversaram na festa após o evento, e logo se apaixonaram. A estadia de
Desmond Abercrombie, antes temporária, tornou-se permanente, e todos os
seus negócios foram transferidos da Escócia para o Brasil.

Vinte e sete anos depois, lá estava ela, moribunda. Sua beleza
abandonara-lhe por completo; já não falava havia certo tempo, pois seu
sistema nervoso fora comprometido pela doença.

Ian observava os olhos vazios de sua mãe, e permitiu-se derramar uma
lágrima. Não havia mais ninguém no quarto; seu pai, o mais desolado de
todos, dormia sob o efeito de um feitiço que seu cunhado, Leonardo,
lançara-lhe; Alden, o irmão mais velho de Ian, falava ao telefone com
diversas pessoas, tentando tocar como podia as Indústrias Abercrombie; e
Leonardo tentava inutilmente preparar mais uma poção que restabelecesse
a saúde de Laura.

Ian ressentia a presença de seu tio. Fora ele o causador do estado
vegetativo de sua mãe, após lançar mão de magia negra para extirpar-lhe
o câncer que estava espalhado por todo seu corpo. O feitiço não
funcionou como deveria, e desde então ele tentava encontrar outras
formas de salvá-la.

Era exatamente por causa daquele tipo de arrogância que Ian odiava
magia. Não bastassem todas as chateações que sofrera com Alden durante a
infância e adolescência, fruto de brincadeiras e pequenos tormentos
causados por feitiços e maldições, ainda havia aquilo. Magia nunca era
usada para nada de útil, exceto destruir o que era puro.

Ian resolveu, naquele instante, que não faria mais parte daquele mundo.
Não importava mais a tradição de sua família. Sua mãe, a única pessoa
que ele realmente respeitava no clã, estava praticamente morta, e sua
vontade de ver o filho seguindo os passos de seus antepassados já não
importava. Ele honraria a memória dela de alguma outra forma, sendo
decente e bom para todo o mundo, não só para os parasitas da Ordem de
Hermes, a entidade que controlava as atividades mágicas no país.

Ciente de que era apenas uma questão de tempo até que ela desse seu
suspiro final, Ian foi até a penteadeira de sua mãe e remexeu nas
gavetas. No fundo de uma delas, encontrou as joias que ela tanto
prezava. Pegou um colar de prata como lembrança. Foi até sua mãe e,
chorando, beijou-a na testa.

--- Desculpa\ldots\,Não posso ficar como o Alden. Sei que a senhora
entenderia. --- Ian passou sua mão trêmula pelos cabelos escuros de Laura,
vertendo ainda mais lágrima e soluçando. --- Eu te amo, mãe. Sempre vou te
amar.

Ian enxugou os olhos e saiu. Foi até seu quarto e colocou algumas trocas
de roupa, o colar e o dinheiro de várias mesadas que juntara em uma
mochila. Tinha em sua cabeceira uma foto dele com a mãe. Enfiou-a no
bolso externo de sua jaqueta. Sem ninguém perceber, desceu as escadas
que levavam para o térreo de sua casa de três andares e de lá saiu para,
assim esperava, nunca mais voltar.

\espaco
\data{São Paulo, 21 de dezembro de 2003}

Érico Porto, de terno preto e com uma rosa vermelha nas mãos, estava
embriagado no funeral de sua mãe.

Um dia antes, a chamada do Jornal das Oito dizia:

  \begin{quote}
  Judite Porto, eminente juíza do Tribunal Superior de São Paulo, morre
  em um trágico acidente de carro.
  \end{quote}

O choque de ouvir a notícia pela televisão antes de qualquer outra
pessoa contar-lhe foi o que desencadeou seu estado ébrio. Érico percebeu
que estava irremediavelmente sozinho, sem ter ao menos alguém para ir
até ele avisar que sua mãe tinha morrido. No dia seguinte, dois
assistentes dela vieram trazer-lhe diversos documentos que precisavam
ser assinados e ofereceram suas condolências, mas as palavras vazias não
lhe serviram de conforto.

Desconsolado, foi até o gabinete de bebidas. Acabara de completar
dezoito anos, e sua mãe comprara-lhe um uísque Abercrombie da mesma
idade para celebrar a ocasião. Segundo ela, essa era uma tradição da
família Porto, e deveria honrá-la assim como os filhos de Érico um dia
honrariam.

Ele havia, na ocasião, tomado apenas um gole. Não gostava de uísque, ou
de bebidas alcoólicas em geral. Naquele dia, porém, precisava mitigar
sua dor de alguma forma, aliviá-la com o que tinha em mãos: a garrafa
estava lá, quase imaculada. Érico removeu a rolha que a fechava e tomou
um longo gole no gargalo. O gosto era terrível e lhe queimava a
garganta. Mesmo assim, de gole em gole, bebeu até desmaiar.

No dia do funeral, com uma dor de cabeça latejante ao acordar, Érico
considerou nunca mais fazer aquilo. Lavou o rosto, tomou uma grande
quantidade de água e foi até o escritório de sua mãe para ligar para as
pessoas que ela conhecia e convidá-las para o funeral. Havia inúmeros
nomes na agenda, mas não sentia conexão com nenhum deles. Devido à
natureza do trabalho se sua mãe, poucas amizades reais eram feitas e
Érico não conseguia pensar em ninguém para telefonar. Ligou para a
secretária que trabalhava para ela, e pediu que ela convidasse quem quer
que fosse apropriado.

Quando pôs o telefone de volta no gancho, chorou. Chorou pois estava
sozinho, seu grande exemplo pessoal e profissional se fora, A mulher que
o inspirara a fazer sempre o melhor que pudesse para os outros, que o
ensinara que era possível salvar o mundo, estava morta e nunca mais o
aconselharia, nunca o veria de beca na formatura, nunca o levaria ao
altar.

Sem pai ou familiares próximos, Érico estava sozinho. E sua perdição
estava prestes as começar.

\espaco

Depois do funeral, Érico continuou bebendo muito, diariamente, além de
medicar-se com os calmantes que sua mãe tomava periodicamente por causa
de todo o stress que seu dia a dia implicava.

Ficara sabendo que ela havia deixado uma quantia enorme de dinheiro como
herança, e por isso não teria muitas preocupações por um bom tempo. Sua
vida passou a ser um borrão, sem nenhuma atividade ou ocupação que lhe
obrigassem a distrair a mente.

Não foi difícil convencer os psiquiatras que estava em profunda
depressão. Estava mesmo, com a perda a única pessoa que realmente amava,
mas apenas procurara os profissionais para conseguir receitas de mais
remédios que abasteceriam seu entorpecimento.

Por conta de um senso de preservação legal que lhe fora imbuído desde
cedo por sua mãe, não procurou conforto em drogas mais pesadas ou
ilícitas. Os calmantes e álcool, no entanto, tinham um efeito
devastador. Perdeu peso, mal tomava banho e quebrou cinco ossos do braço
por acidente caindo no banheiro enquanto estava bêbado. Tudo parecia
perdido; não havia luz no fim do túnel para ele. Mas então veio o sonho,
e tudo começou a mudar.

Érico estava um dia caído com uma garrafa de vodca Absolut aberta na
mão, babando profusamente no sofá de couro. No meio de seu sono
profundo, Érico viu-se consciente durante um de seus sonhos. Ele estava
em uma cela feita com pedras e cuja porta e janelas eram feitas de ferro
mal trabalhado. Parecia uma prisão medieval, e Érico estava desesperado.
De repente a porta se abriu, e ele caminhou para o corredor. Lá, não
encontrou o que esperava: ao invés de outras celas, havia um
triceratopo, o dinossauro favorito dele quando era criança. Seu sonho,
na época, era tornar-se paleontólogo.

Uma jovem e atraente mulher, vestindo uma saia rodada e uma blusa de
alcinhas, jogou seus cabelos cacheados para o lado e começou a acariciar
a cabeça da criatura. Aquilo não era verossímil, pois triceratopos
tinham o tamanho de vários ônibus, não de uma vaca mas, como todo o
resto não fazia sentido também, apenas deu de ombros. A mulher mirou-o
com seus olhos verdes e disse com a voz mais doce que tinha ouvido:

--- Vá achar seus dinossauros, garoto.

Ela então se dissolveu em um monte de areia em um torvelinho frenético,
e a consciência de Érico foi arrastada junto até ele acordar.

Desperto, ele olhou ao redor e pela primeira vez em muito tempo percebeu
o que estava acontecendo. A garrafa ainda estava em sua mão, o
apartamento imundo e sua vida uma bagunça. Levantou-se com dificuldade e
tomou o primeiro banho do resto de sua vida.

Uma semana depois, estava matriculado em cursinho de vestibular, limpo e
sóbrio. Aplicou-se, lutou contra as crises de abstinência que o
acometeram e procurou ajuda de verdade.

Com seis meses de estudo, conseguiu passar no primeiro curso de
Paleontologia da Universidade Estadual de São Paulo em Franca, e para lá
se mudou. O sonho de salvar o mundo lentamente voltou e, embora não
tivesse ninguém para compartilhá-lo, sabia que estava mais perto então
de realizá-lo do que nunca.
