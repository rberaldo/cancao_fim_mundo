\chapter{Muito Prazer, Eu Sou Você Amanhã}

\data{Franca, São Paulo \\ 
Maio de 2005}

A rotina de Érico, desde que decidira retomar o objetivo de sua infância
e tornar-se paleontólogo, era a mais espartana possível; acordava
bastante cedo, corria, tomava um café-da-manhã balanceado e estudava.
Nada mudara desde seu ingresso na universidade, apesar da
incompatibilidade de seus hábitos com os de seus colegas calouros.
Continuava avesso a festas, à bebida e mesmo a relacionamentos.

Antes da morte de sua mãe, Érico mantinha diversos affaires com meninas
de sua idade, além de ter outros amigos pouco próximos. Seu círculo
social diminuíra drasticamente desde então, e levava em Franca uma vida
solitária.

Ele estava no único quarto de sua quitinete lendo um livro de
mineralogia quando recebeu uma ligação. Uma de suas colegas de sala
convidou-o para ir a uma festa, mas ele respeitosamente declinou, embora
tivesse ficado tentado. Não achou, porém, que teria forças para chegar
perto de álcool novamente, ainda mais em um ambiente extremamente
propício para seu consumo. Temia acordar seus demônios, com os quais
lutava por quase dois anos.

Não, não iria esmorecer. Apesar de um pouco decepcionado com o curso,
teórico demais, e com seu isolamento auto-imposto, Érico ao menos tinha
um propósito e estava se esforçando para alcançá-lo. Consciente disso,
voltou à sua leitura que, mesmo enfadonha, o ajudaria em seu caminho.

Embora extremamente concentrado na tarefa que se propusera a fazer,
ouviu a porta da cozinha que dava para fora abrir com um leve rangido e
levantou-se de supetão. Não fora o vento que provocara aquilo, sabia
disso. Largou o livro e assumiu uma posição de combate vista em um velho
filme de ninjas, esperando pelo invasor que fatalmente entraria no
quarto.

E ele entrou, mas não era o que Érico esperava. À sua frente estava um
homem vestindo um jeans surrado, coturno preto e uma camisa xadrez de
flanela, com o rosto levemente marcado por cicatrizes, sendo a mais
marcante em formato de “S” bem em sua testa. Usava uma barba cheia, com
alguns fios grisalhos que pontuavam-na, assim como seu cabelo. Embora
parecesse castigado pela vida, seus olhos brilhavam com jovialidade.

O homem era ele mesmo

--- Olá, eu sou você em 2013. O mundo não terminou ano passado, como
você pode ver. --- disse, referindo-se à famosa profecia maia. Seu eu
futuro sorria largamente. --- Não ligue para essa cicatriz, não vai doer
tanto assim.

Apesar de toda a estranheza que a situação causava, não pensou em outra
coisa a dizer:

--- É\ldots\,É chato saber que eu terei uma cicatriz em forma de S.
Preferia um raio.

--- Imagino que sim. Sei que deve ter muitas perguntas, mas você precisa
saber de uma coisa, e apenas uma, agora: vá para o fim da Rua Elmo.
Imediatamente.

--- Para que? Por quê?

--- É lá que você tem de encontrar o Doutor.

--- Doutor?

--- Exato. É ele que tornará tudo possível. E está vindo apenas para te
encontrar. Ouça, --- disse, apontando um dedo para seu eu do passado ---
não desista. Nunca. Lembre-se do sonho.

--- Você está dizendo que eu preciso achar meus\ldots\,Dinossauros? Mas
eu já estou aqui, fazendo Paleontologia e\ldots

Seu eu do futuro apenas olhou para ele. Seu semblante era inescrutável.
Virou-se e saiu da casa. Quando conseguiu vencer o choque, Érico correu
atrás dele, mas não havia mais nada.

Fora apenas um sonho? Uma alucinação? Érico desabou em uma cadeira da
cozinha, checando se estava com febre. Temia também que aquilo fosse um
subproduto de seus velhos hábitos com calmantes. Suas indagações
cessaram. Pensou na ordem dada pelo Érico de 2013. Mesmo que tudo fosse
apenas uma peça que sua mente lhe pregara, que mal haveria em segui-la?

Estava frio em Franca. Érico apanhou sua jaqueta, vestiu-a e saiu de sua
casa em direção a seu destino.
