\chapter{Doutor Lloyd}

No fim da rua Elmo, o tempo parou. Érico notou que a poeira visível por
um facho de luz, irradiado da lâmpada que vinha do poste, tinha ficado
estática. Não havia mais quaisquer barulhos, exceto os que ele mesmo
fazia em suas interjeições de surpresa e estupefação. Olhou ao redor:
uma aranha que subia pela parede já não se movia mais, a luz de uma
torre de televisão ao longe não piscava como outrora. Em seu esforço de
entender o que ocorria, não percebeu um pequeno barulho que rompeu
momentaneamente o espaço-tempo.

Uma voz animada e um tanto confusa soou atrás dele:

--- O camp-de transção espas-temporal permit ki as-ondas-onoras-i propag
normente, ki vcê respire. O temparou não; tá extremente plexado, lento.
Seu cébro gistra velcidade minma não, entã prece que tudistá dês-jeito,
cogelado. --- disse o homem, que trajava um jaleco branco por cima de uma
calça preta feita de um tecido fosco desconhecido. Alto, tinha cabelos
loiros platinados, e olhava para Érico com uma espécie de admiração e
enternecimento. Uma moto de corrida preta estava logo atrás dele.

--- Quem é você? --- perguntou Érico, boquiaberto tanto com o evento como
com a estranha variação de português que o homem utilizava.

--- Me no-mé Dr. Lloyd. Sô do futuro, tem-o algo p´cê. Descul-pa fala.
Fala-os no futuro asi. Que-ro ki vcên-tendi. Es-é Kamen, --- disse ele,
apontando para o veículo --- um-oto ki é maqna du-tempo. Ma-só páp-sado.
Nuca futuro. Sim?

--- Desculpe, não estou entendendo você. --- disse ele. Embora o homem a sua
frente falasse com fluência e certo cuidado, muitas coisas faltavam, o
que dificultava a compreensão.

Dr. Lloyd aproximou-se de Érico, que por sua vez deu dois passos para
trás.

--- Calma. Vô mach-cá vcê não. P´cê, uma-qna do tempo. Só passado, sim?
Viagem só páp-sado.

--- Não é só isso, quero saber o que está acontecendo.

--- Vcê pici-savá mundo. Eu dô maqna p´cê savá. Ma-santz, vá-chá seus
dinossauro, garoto. --- disse ele, estendendo a chave da moto para Érico.

--- Como isso é possível?

--- Só páp-sado, lembra. Tem-o kim bora. Sote.

O doutor apertou um pequeno botão em seu relógio e instantaneamente
sumiu, fazendo com que o tempo voltasse ao normal. A aranha continuou em
seu caminho, a luz ao longe piscou como de praxe e a poeira movia-se com
o vento.

Érico olhou para a moto, que ainda estava lá, e se aproximou para
tocá-la. O frio do metal em contato com sua pele convenceu-o de que
aquilo não era um sonho, apenas um acontecimento muito, muito estranho.

Talvez fosse verdade que seu eu do futuro tinha aparecido e que,
momentos antes, um homem do futuro tinha lhe entregado uma máquina do
tempo em forma de moto para, pelo que entendera, salvar o mundo. Só
havia um jeito de provar a verossimilhança de tudo: testar sua Kamen.

Sentou-se no banco, segurou o guidão esquerdo e colocou a chave na
ignição, virando-a. Uma tela apareceu no lado de dentro da bolha, com a
inscrição: “Érico Porto: entrar digital”. Havia um pequeno quadrado
iluminado, do tamanho certo de um polegar. Pressionou-o na tela, e uma
vez computadorizada disse-lhe: “Bem-vindo, Érico. Para quando gostaria
de ir hoje?”
