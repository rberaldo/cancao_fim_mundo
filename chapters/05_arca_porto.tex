\chapter{A Arca de Porto}

Érico começava a questionar sua sanidade. A moto que recebera do Dr.
Lloyd ligava, podia ser dirigida por toda cidade e seu computador de
bordo indicava os melhores caminhos graças ao sistema de posicionamento
global nele instalado. No entanto, não havia uma pista sequer de que
aquela era uma máquina do tempo.

O computador perguntara-lhe para quando ele gostaria de ir. Ele sabia o
que ouvira. Ou sabia? Naquele instante, nada mais era certo. Averiguara
sua conta bancária e percebeu que, dias antes, gastara uma quantia
razoável de dinheiro em uma concessionária de motos.

Aquela noite, a que supostamente mudara tudo, fora uma mentira. Ele
mesmo tinha comprado a moto e se convencido de que era especial,
lançando mão de uma alucinação para isso. Restava-lhe então saber o que
ocasionara aquele lapso. Em seu período de luto, aquele em que se
entregara à bebida e aos calmantes, por muitas vezes não consegui
distinguir seus devaneios da realidade. Só que ele não bebia desde que
tivera aquele sonho que o fizera mudar de cidade, de atitude, de vida.
Ou pelo menos era o que ele pensava.

Talvez, e aquela era uma ideia que o aterrorizava mais do que qualquer
outra, estivesse bebendo e não se lembrasse.

Estava em seu quarto quando pensou naquilo, deitado na cama. Levantou-se
e abriu freneticamente as portas de seu guarda-roupa embutido, revirando
tudo em busca de alguma garrafa ou cartela de remédios que pudesse ter
obtido sem saber conscientemente. Nada. Em seguida, foi até a cozinha e
ao banheiro, os únicos outros cômodos de sua quitinete. Também não vira
nada além de utensílios, produtos de limpeza e embalagens de comida
congelada.

Sentou no chão frio da cozinha. A angústia o atormentava. Não podia ser,
ele não podia ter imaginado aquilo. Lembrou-se daquele que julgara ser
seu eu do futuro, que lhe dissera que deveria continuar acreditando, que
deveria se lembrar da promessa de encontrar Ian naquele bar nos próximos
meses. Por que? Sentia uma falta imensa de seu amigo, da vida mais
simples que levavam na época do colegial, mas por que seu subconsciente
falaria aquilo?

Érico pensou, então, no que faria se pudesse ligar a máquina do tempo.
Já que não podia ir para o futuro, voltaria para aquela manhã em que se
viram pela última vez. Diria tudo que não dissera, pois pensava então
que Ian não estava de fato se despedindo, que haveria um adeus
apropriado.

Foi então que ouviu um barulho vindo de fora. Eram pequenos beeps
digitais, emitidos pelo computador de bordo da Kamen. Enxugou as
lágrimas que brotavam de seus olhos e saiu.

Aproximou-se da moto preta. A tela estava ligada, e uma frase escrita em
letras grandes chamou sua atenção: \textsc{sistema de viagem cronal
preparado}.

Tocou no cristal líquido, esperando que a mensagem se apagasse, que tudo
fosse uma nova alucinação. Ouviu, porém, a mesma voz de antes:

“Senhor Porto, o sistema de viagem cronal está por fim estabilizado após
receber ajustes de obtenção de energia. O motor agora processa luz solar
para abastecer as baterias. O uso de gasolina, além de prejudicial ao
seu meio ambiente, se faz desnecessário. Para quando gostaria de ir
hoje?”

“Como\ldots?”

“O sistema teve de ser atualizado após a última viagem cronal, por isso
foi desligado abruptamente.”

“Isso não está acontecendo, não está\ldots” --- disse ele, mais para si
mesmo do que para o computador.

“A veracidade dos acontecimentos pode ser atestada por inúmeras equações
matemáticas, senhor Porto.”

“Eu vi que eu comprei a moto!”

“Um arranjo feito pelo senhor, para o caso de perguntas serem feitas.”

“Eu não lembro de ter feito nada.”

“É porque o senhor ainda não fez.”

“Eu não fiz? Ainda!?” --- perguntou ele, irritado com as respostas
monocórdias da máquina.

“O ato em si foi efetuado há duas semanas. O senhor, vindo de dois mil e
treze, o fez.”

Érico então começou a entender, mas não se deu por satisfeito. Para ele,
aquilo ainda era uma alucinação.

“Ok, vamos viajar, então. Que tal ir para a porra do dia em que os
dinossauros foram extintos?” --- disse, zombeteiramente pulando na moto,
crente de que acordaria ou que aquela visão, o que quer que fosse,
desapareceria por completo se ele a confrontasse.

“Pois não.” --- disse o computador cortesmente.

Houve um estrondo surdo, um clarão, e repentinamente tudo ao seu redor
estava mudado. O estacionamento, os carros e motos, tudo se foi; uma
mata cheia de árvores altas cercava a clareira onde ele agora estava.

“O campo de contenção impede que haja contato físico entre você e o
mundo exterior, para que se evite contaminação.”

Érico pensou, por um instante, em perguntar do que o computador falava,
mas estava maravilhado demais para produzir qualquer som, para elucidar
qualquer dúvida. A vegetação era diferente de tudo que já vira em
livros, muito florida e havia árvores que serviam claramente como fonte
de alimentação para alguma criatura\ldots

E então ele viu, sentindo o chão tremer, um tricerátopo se aproximar a
passos largos e calmos. Sua pele era rugosa e marrom, coberta de pêlos
fios e longos. Os chifres eram um pouco amarelados, os olhos grandes e
castanhos; de sua boca pendia um fio de saliva espessa. Aquela era a
coisa mais linda que Érico vira em toda sua vida.

A criatura passou ao seu lado, completamente alheia à sua presença. Ele
novamente começou a sentir seus olhos lacrimejarem, por outro motivo
daquela vez. Pôs-se a ouvir os sons, a se deleitar com a visão das cores
das plantas e dos enormes insetos que voavam para cá e para lá. Não
importava mais que aquilo não pudesse ser real. Queria absorver o máximo
daquela experiência, poder dizer aos seus colegas que tricerátopos
comiam plantas rasteiras e flores\ldots

Novamente o chão tremeu. Érico esperou, maravilhado, que alguma outra
criatura aparecesse. Em vez disso, o céu se iluminou com uma rajada de
fogo e um barulho ensurdecedor castigou os ouvidos de todas as
criaturas, que se lançaram ao desespero. Aquele era o fim.

Érico não sabia o que fazer. Desesperado, interpelou o computador:

“O que diabos está acontecendo?”

“O senhor ordenou que viéssemos até o dia em que os dinossauros foram
extintos pelo meteoro.”

“Não\ldots\,Eu só queria\ldots\,Tem algo que eu possa fazer?”

“Não.”

A terra tremia cada vez mais, e ondas de calor podiam ser sentidas mesmo
através do campo de contenção. Érico não podia presenciar aquilo, e
ordenou à Kamen que fossem embora.

Estavam de volta ao estacionamento. Ele tremia em cima da moto, suas
mãos agarravam com força os dois guidões.

“Isso foi real, não foi?”

“Sim, senhor.”

“Não estou mesmo tendo alucinações?”

“Não, senhor.”

“Como você sabia para quando ir, exatamente?”

“Meu banco de dados é do seu futuro. Isso já foi descoberto no
século~\textsc{xxiv}.”

“Muito bom. Pelo que eu entendi, você pode me levar a qualquer lugar do
tempo e do espaço, com exceção do futuro.”

“Desde que saiba para onde e quando quer ir.”

“E que lugar era aquele?”

“A ilha de Jira, senhor.”

“Eu poderia, em tese, trazer coisas do passado para cá?”

“Sim, senhor.”

“Pois bem, Kamen, vamos trazer uns dinossauros para nossa época antes
que eu comece a salvar o mundo.”
