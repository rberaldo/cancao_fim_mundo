\chapter{O Gueto Ratskillz}

Ian entrou clandestinamente no primeiro navio que encontrou no porto de
Santos. Não podia viajar com seu passaporte, pois assim seria localizado
por seu irmão ou por seu pai, que com certeza iriam procurá-lo.

O Falcão do Milênio levou-o até Lisboa. Pela primeira vez livre na vida,
Ian aproveitou a noite portuguesa, tomou bons vinhos, comeu bons queijos
e, em um curto espaço de tempo, gastou praticamente todo o dinheiro que
levara. Começou a trabalhar em restaurantes e postos de gasolina em suas
viagens por diversos países da Europa para ganhar uns trocados, ficando
sempre pouco tempo em cada emprego para não correr nenhum risco de ser
achado.

Seus cuidados também causavam outros inconvenientes. Como a maior parte
dos hotéis e albergues europeus pede passaporte para oferecer
hospedagem, Ian recorria a sua barraca para dormir nos parques e campos.
Foi, por conta disso, inúmeras vezes assaltado e até esfaqueado em uma
ocasião.

Ian passou um ano fazendo isso. Chegou a Londres em~2005, pronto para
seguir com sua rotina errante. Iria ficar lá por no máximo duas semanas
e depois seguiria para a França. Escócia, nunca.

Como de praxe, ele tentou achar um quarto na periferia da cidade, longe
das atenções das autoridades ou dos círculos que seu irmão ou pai
pudessem frequentar. Achou um flat sem burocracias e alugou-o por um
mês, embora fosse ficar bem menos tempo que isso. Esse era seu modus
operandi: só ficava em lugares que não pedissem documentação, saindo
após poucos dias para evitar riscos. Uma saída cara, mas razoavelmente
segura. E, depois de passar três semanas dormindo em sua barraca em
pleno inverno, achou que merecia um pouco de conforto.

Já estava escuro. O flat ficava em um gueto, e Ian gostou da aparência
do lugar: sujo, perigoso e caindo aos pedaços. A polícia provavelmente 
não apareceria. Sorrindo, entrou no quarto-e-sala, pôs a velha mochila 
no sofá-cama e estirou-se, caindo imediatamente no sono.

Ian sonhava com uma bela norueguesa que conhecera na Cracóvia quando
sofreu um rude despertar. Uma força colidiu com a porta do flat, e havia
fogo do lado de fora. Imediatamente alerta, ele, que dormira com toda a
roupa, levou a mão à bota esquerda, e de lá tirou uma faca.
Posicionou-se atrás da cama, à espera de alguma ameaça.

Uma briga tétrica ocorria no exterior. Ian ouviu gritos e, para seu
temor, uma encantação em latim que ele conhecia bem. Olhou pelo flat, e
desesperou-se pois a única janela do lugar ficava exatamente ao lado da
porta. Houve uma explosão que fez o chão tremer e depois silêncio.

Depois de alguns minutos, ainda não havia comoção. Ian decidiu dar uma
espiada no que estava acontecendo. Foi nas pontas dos pés até a janela e
viu pelo vidro dois homens: um em posição de combate, sem se mexer; o
outro esbofeteava-lhe as faces, como se estivesse dormindo.

Ian deduziu o que ocorrera e saiu do flat. Esgueirou-se pelo fogo que
crepitava defronte sua porta e se aproximou dos dois. O que se movia
estendeu seu braço e gritou em inglês:

--- Afaste-se!

Ian levantou os braços indicando que não carregava nada (a faca voltara
para sua bainha na bota) e disse calmamente em seu próprio inglês,
moldado à perfeição devido à convivência com o pai e anos passando o
verão na Escócia:

--- Não sei o que está acontecendo, mas posso imaginar. Vocês foram
atacados por magos, e seu amigo está paralisado.

--- E o que isso tem a ver com você, hein?

--- Bem, conheço um jeito de reverter a paralisia.

Antes que pudesse dizer qualquer outra coisa, Ian foi telecineticamente
suspenso no ar. Sentiu sua garganta apertar um pouco. O homem a sua
frente estava causando aquilo, pois o movimento de sua mão antecedeu em
milésimos de segundo as ações sofridas.

--- Quem é você? --- vociferou o homem, ensandecido.

--- Não importa meu nome; se seu amigo não tiver o feitiço revertido
logo ele irá parar de respirar --- disse, engasgando.

Considerando isso, o homem abaixou a mão, fazendo com que Ian caísse. O
jovem Abercrombie levantou-se e foi até ambos.

--- Se você tentar algo, juro que te mato.

Ian nada disse de volta, apenas murmurou:

--- \textit{Soluo corpum}.

O homem paralisado instantaneamente voltou a se mexer, arqueando seu
corpo em tosses longas, buscando o ar. Auxiliado pelo outro, disse após
certo tempo:

--- Valeu, mano. Valeu mesmo.

Ian assentiu com a cabeça e virou-se, pronto para voltar para o flat,
pegar suas coisas e dar o fora. A voz daquele que quase o estrangulara
parou-o, porém.

--- Ei, volte aqui! --- gritou, mas já sem hostilidade. --- Sério, quem
é você?

--- Michael. Meu nome é Michael. --- disse Ian, virando-se. Essa era sua
alcunha nas viagens.

--- Bem, Michael, me desculpe pelo pequeno\ldots\,ah, incidente de
agora. Não sabia que você era um dos bonzinhos.

--- O que? --- perguntou ele, confuso.

--- Um dos bonzinhos. Magos bonzinhos.

--- Não sou mago. --- corrigiu ele apressadamente.

--- Mas como você\ldots

--- Meu\ldots\,ah, amigo fazia isso comigo. É por isso que eu sei que
depois de alguns minutos o ar começa a faltar, bem como a encantação
para reverter o feitiço. Esse é um dos únicos que podem ser revertidos
com apenas palavras. Qualquer um pode fazê-lo. É só dizer soluo corpum.
É latim para “eu liberto o corpo”. Enfim --- disse desconfortavelmente
--- adeus.

--- Calma, rapaz. Meu nome é Peter. Peter Stone. Sou um Estranho.

--- Bem, sei disso. Não nos conhecemos, e espero que continue assim. ---
disse Ian friamente.

--- Não, você não entendeu --- insistiu Peter --- sou um Estranho. Um
mutante, meta-humano, seja lá o que chamam hoje em dia. Tenho poderes,
super-poderes. É por isso que eu consegui\ldots\,ah\ldots\,levantar você
e.\ldots\,--- disse, gesticulando com certo embaraço e culpa. --- Eu
achei que você fosse da turma do Dedalus. Este é Charles. --- apontou
para o homem de aparência jovem. --- Ele deve a vida a você.

Charles estava aparentemente pensando em alguma outra coisa, e nada
disse. Tanto Ian quanto Peter olharam para ele, o primeiro mais
apreensivo que o segundo. Finalmente Charles se pronunciou:

--- O nome do carinha não é Michael. E ele não é daqui.

--- Devo me preocupar? --- perguntou Peter, prestes a levantar sua mão
direita novamente.

--- Não, ele parece ser realmente maneiro. Nada a ver com Dedalus. Gente
fina e sangue bom.

--- Como você\ldots

--- O talento dele é ler mentes. Isso é que é interessante sobre os
Estranhos; eles só têm um talento, inato e imutável. E você, Ian, é um
de nós?

--- Absolutamente não. Minha única estranheza foi ter aprendido essa
encantação e só. --- disse, se esquivando. Não queria ficar mais lá,
pois estava se expondo cada vez mais.

--- Você odeia magia, como a gente. --- disse Charles. --- Essa turma só
causa merda. Vem aqui no nosso território dizer o que rola a gente
fazer, o que não rola, fala que tem que pagar uma merreca para eles
senão eles não deixam a gente em paz e essas paradas. Só filho da puta.

--- Dedalus é um representante desgarrado da Royal Academy of Sorcery, e
comanda uma trupe de magos. De vez em quando eles aparecem aqui em
Ratskillz para nos importunar. Essa foi a mais branda, para falar a
verdade. Perdemos dois amigos ano passado, e tenho certeza que foram
eles.

--- É muito bom ficar avisado disso. Vou tomar cuidado. --- disse Ian,
querendo se afastar. Deu dois passos para trás, mas Charles agarrou-o
pela manga.

--- Você tem que ficar com a gente, Ian!

Ian estava aterrorizado com aquilo; era primeira vez que alguém o
chamava pelo nome em mais de um ano. Desvencilhou-se de Charles e foi se
afastando.

--- Seu nome é Ian, então. --- disse Peter, tentando apaziguar as
coisas. Sua voz calma e profunda fez com que Ian parasse. --- Charles me
diz por nosso elo telepático que você tem percorrido a Europa, sempre
fugindo. Por que não para? Por que não fica aqui, conosco?

--- Desculpe, não estou interessado. Olhe, preciso ir.

--- Espera, Ian! --- gritou Charles. --- Se não rola se juntar a gente,
pelo menos fica aqui em Rastkills. Você vai ficar seguro, ninguém vai te
procurar aqui. E os caras do Dedalus não voltam tão cedo assim, pode
ficar sussa.

Ian, apesar de ter dormido bastante, ainda estava cansado, e Charles dia
a vida a ele afinal. Talvez pudesse ficar mais um dia para descansar.

--- Ok, vou ficar. Mas não posso me juntar à sua\ldots\,Turma.

--- Somos os Estranhos de Essex. --- corrigiu Peter. Em seguida,
estendeu a mão em direção ao fogo defronte ao flat de Ian e, com um mero
abano, o ar esfriou. As chamas se apagaram instantaneamente.

Ian ficou curioso. Peter alegara ter apenas um poder. Como mover objetos
com a mente e congelar coisas poderiam contar como apenas um? Perguntou
isso a ele, que respondeu:

--- Sou um mímico; tenho a habilidade de copiar os poderes de outros
Estranhos.
