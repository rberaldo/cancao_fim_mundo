\chapter{Os Estranhos de Essex}

Ian, Peter Stone e Charles Essex estavam no meio da rua, pouco depois do nascer
do sol. A maioria das casas do Gueto Ratskillz estava abandonada, então a luta
de minutos antes causara pouquíssima comoção.

--- Você imita habilidades? --- perguntou Ian a Peter.

--- Exato. O que outro Estranho faz, eu sou capaz de fazer. Uma coisa de cada
vez, é verdade, mas a ideia geral é essa. Posso mover as coisas com a mente ---
desculpe de novo por aquilo, aliás --- congelar coisas, ler a mente dos
outros\ldots\,bem, deu para entender. E faço tudo naturalmente, quer dizer, sem
ter de aprender encantações, palavras mágicas e essa merda toda. Requer
esforço, claro, mas é como aprender a jogar bola.

Ian não sabia o que dizer. Durante toda sua vida, teve ciência de que coisas
fantásticas existiam no mundo, coisas das quais ele se ressentia devido às
experiências que tivera com sua família. Aquilo, porém, era algo novo e, para
seu espanto, bem-vindo. Os homens à sua frente tinham habilidades únicas, eram
humildes, pareciam ser de boa índole e, acima de tudo, odiavam magia como ele.

Peter telecineticamente removeu os detritos que estavam espalhados pela rua, e
os amontoou em um dos terrenos baldios, ao lado do flat que Ian alugara. Ao
terminar, voltou-se para os outros dois.

--- Charlie, vou para casa. Fique aqui com Ian. Vou buscar seu pai para
conhecê-lo.

Charles assentiu com a cabeça, e Peter desapareceu subitamente.

--- O que foi isso? --- perguntou Ian, novamente surpreso.

--- O Peter faz essa parada de sumir e aparecer em outro lugar. Os caras lá
naquela série de nerd fazem isso. Acho que é Guerra nas Estrelas.

--- Jornada. --- corrigiu-o Ian.

--- Isso, Guerra na Jornada. Com aquele elfo, o Spock.

Fã de ficção científica, Ian ficou ofendido com a completa ignorância de
Charles, mas resolveu deixar passar. Havia questões mais prementes a serem
tratadas.

--- Peter disse que traria seu pai para me conhecer.

--- É, num instante eles tão aqui.

--- Não acho prudente ficar conhecendo pessoas assim, me desculpe. Acho que vou
embora.

--- Olha, meu pai não tem nada a ver com seu bro, a gente nunca ouviu falar
nesse Alden, fica sussa.

--- Pare de ler minha mente! --- gritou Ian.

Charles olhou para ele um pouco chateado.

--- Desculpa, Ian. Não consigo evitar. Mas vou parar, sim, eu te devo minha
vida. É o mínimo que posso fazer.

O jovem Abercrombie ficou desconcertado com seu ato. Não gostava da invasão de
privacidade e da ideia de ter sua mente esquadrinhada por alguém que não
conhecia, mas Charles parecia ter boas intenções.

--- Desculpa ter gritado. Só pare, por favor. --- disse em um tom sincero,
porém firme

--- Beleza. Olha, a gente é firmeza. Talvez você não precise ficar fugindo
mais, pode virar um de nós, mesmo não sendo um Estranho.

--- Sem, ofensa, Charles, mas eu não me interesso por esse tipo de coisa. Como
você pode ter visto --- disse, apontando ironicamente para sua própria cabeça
--- sou avesso a essa merda toda. Além do mais, o que eu poderia fazer além de
dar um soco ou outro em algum mago filho da puta?

--- Fala sério! Você me desparalizou, velho! Eu tava lá, sufocando, e você
veio, falou uma palavrinha em grego e me consertou. Se isso não é ser útil, sei
lá o que é.

--- Foi em latim, --- murmurou ele, constrangido com a crescente ignorância do
rapaz. --- e isso foi uma coincidência. Meu irmão fazia isso comigo o tempo
todo, para me atazanar. Minha mãe disse essas palavras incontáveis vezes antes
de aplicar um belo castigo a Alden para fazê-lo parar. É por isso que as
conheço: coincidência. Nada mais.

Antes que a discussão pudesse continuar, um pequeno barulho desviou a atenção
de ambos. Peter materializou-se novamente, trazendo consigo outro homem, de
meia-idade, que trajava um sobretudo bege.

Sobranceiro, altivo, o pai de Charles se portava com uma elegância ímpar,
analisando Ian discretamente. Concentrou seu olhar por alguns instantes em sua
testa, mas logo o momento passou, e falou com sua voz fluida e pacificadora:

--- Olá, sou Victor Essex. Soube que ajudou a meu filho, e achei que devia
agradecer o gesto.

Victor estendeu sua mão, e Ian apertou-a sem dizer nada.

--- Perdoe-me pela intromissão em seus afazeres, --- continuou --- mas gostaria
de ter sua atenção por mais alguns minutos, Sr.~Abercrombie. Podemos tomar um
chá? Há uma delicatessen muito boa logo ali.

Intimidado com a presença daquele homem confiante e extremamente
bem-edu\-ca\-do, que contrastava com o ambiente quase pestilento ao seu
redor, Ian assentiu e seguiu Victor Essex e Peter. Charles postou-se ao
seu lado, como se a protegê-lo, e todos caminharam até o lugar indicado
por Victor, que ficava e um quarteirão e meio de distância.

A delicatessen, chamada The Wolf’s Lair, era pequena, com apenas cinco mesas em
seu interior. Estava completamente vazia, exceto pela atendente que lia uma
edição do The Sun. Ao ouvir a porta ser aberta, ela imediatamente ergueu os
olhos e largou o jornal, apontando a mão esquerda na direção dos quatro. Ao
reconhecer Victor, relaxou o braço.

--- Ah, Essex, é você. Achei que aqueles punheteiros iam voltar.

--- Causaram algum problema por aqui, Dolores?

--- Eles normalmente não mexem comigo, mas nunca se sabe. Ainda bem que não
havia clientes aqui na hora da briga. E vocês de novo, hein --- disse
severamente, olhando para Peter e Charles. O último baixou a cabeça, chateado,
mas Peter não quis saber de bronca.

--- D., você sabe que eles vieram aqui sem ser chamados. Nós não queríamos
brigar. Quando eles vieram, tentamos conversar, dissemos que esse não era o
lugar nem a hora, mas não nos ouviram.

--- Hmpf --- bufou ela, que voltou a ler o jornal. Peter apenas murmurou algum
palavrão, e foi até uma mesa, sentando-se. Victor apontou a mesa para os outros
dois, que se juntaram a Peter, enquanto ele caminhou até o balcão e entrou na
parte dos fundos, em que estava a cozinha. Curioso a respeito daquilo, Ian
perguntou:

--- Seu pai é dono deste lugar?

Charles parecia concentrado em alguma coisa, então Peter respondeu por ele:

--- Não. Como Dolores é uma de nós, temos acesso livre às dependências daqui
para nos escondermos, comer ou beber algo. Geralmente ela não gosta de fazer o
chá, então o senhor Essex vai até a cozinha e o faz. Ele, aliás, está muito
interessado em você.

--- Não vejo motivo para isso. Sou apenas um cara viajando por aí. --- disse
Ian, tentando dar de ombros.

--- Charlie disse que você foge. --- disse Peter, apontando para sua cabeça, em
menção à comunicação telepática compartilhada por ele e Charles --- De sua
família, de seu irmão, de magia. E ele pode ser burro como uma porta, mas sabe
investigar a mente de uma pessoa e avaliar seu caráter em um instante. É por
isso que o senhor Essex queria te agradecer e conversar pessoalmente com você.

--- Você disse que eram os “Estranhos de Essex”. O que isso significa de
verdade?

--- Significa --- disse Victor, que voltara carregando uma bandeja com quatro
xícaras cheias de líquido fumegante --- que nós nos reunimos em prol de um
objetivo em comum, e que eu sou o líder desse grupo. Não foi ideia minha, na
verdade, mas o uso foi sedimentado o bastante para nos identificar
irrevogavelmente.

A voz de Charles ecoou na cabeça de Ian, rindo:

--- Ele fala assim mesmo, não se preocupe.

Ian ignorou o comentário, e continuou a inquirir Victor, que tomava um gole de
chá:

--- Qual é esse objetivo?

--- Evitar nossa extinção. --- disse, em um tom leve --- Tome o chá, por favor.
Devo dizer, sem modéstia nenhuma, que ele está delicioso.

Ian bebericou um pouco. Realmente estava muito bom. Seu estomago, estimulado
pela bebida, roncou. Não comia nada desde que chegara a Londres, e pensou que
um daqueles bolinhos cairia muito bem.

--- Pai, --- disse Charles, levantando-se --- vou pegar uns bolinhos para nós.

Ele virou a cabeça, e deu uma piscadela para Ian, que sorriu, grato. Victor
pigarreou levemente, e continuou:

--- Quando digo que estamos sendo extintos, não é mera força de expressão.
Estranhos desenvolvem habilidades sobre-humanas espontaneamente, sem as
restrições que a magia impõe a seus praticantes. Poucas pessoas aceitariam
nossa existência de bom grado. Nossa rixa com os magos é apenas parte do
problema; equiparamo-nos em nossos embates. Quando digo que estamos sendo
extintos, refiro-me aos estranhos desaparecimentos ou assassinatos de pessoas
de nossa espécie.

“Semana passada, dois jovens foram brutalmente assassinados em Manchester. Eles
eram irmãos, e eu havia conversado com ambos há um mês. Eu os alertei, assim
que soube de sua existência, dos perigos de mostrarem suas habilidades a outras
pessoas. Sei que me ouviram, e mesmo assim foram mortos, misteriosamente.

“Poderia ser um incidente isolado, claro. Três meses atrás, porém, uma jovem
chamada Nellie foi morta em sua casa. Estrangulada. Nós estávamos observando-a,
esperando o momento certo de abordá-la, pois ela ainda não manifestara suas
habilidades. A espera revelou-se ineficaz; poderíamos ter salvado sua vida.

“Minha teoria, senhor Abercrombie, é que alguém tem feito isso deliberadamente.
Pode ser algum membro da Royal Academy Of Sorcery agindo por conta própria, ou
algo pior.

--- E por que os magos também não têm sido mortos?

--- Os magos já têm seu estrato consolidado. São, de muitas maneiras,
intocáveis. Incomodá-los seria imprudente. Já nós, Estranhos, estamos muitas
vezes solitários, desorganizados. Nosso grupo é uma das poucas exceções, e
talvez por isso não constituamos um alvo fácil.

--- Como ninguém mais sabe disso?

--- Senhor Abercrombie --- disse ele, muito sério --- Sabendo da
existência de magos e de outras criaturas fantásticas, não acha estranho
que \emph{nada} disso seja revelado ao grande público? O anonimato pode
ser o alicerce da manutenção de nossas atividades, mas pode muito bem
ser também a ferramenta usada para nossa destruição.

--- Desculpe, não entendo.

--- Deveria haver pessoas voando no céu. --- disse ele, apontando para cima ---
Deveria haver magias sendo invocadas para consertar pontes, salvar vidas e
proteger as pessoas. E nada disso está acontecendo. A quem interessa todo o
silêncio, todo o segredo? A resposta é simples: Àqueles que querem nos
destruir. E é por isso que acho que você deve se juntar a nós.

--- Desculpe, mas não sei do que está falando. Peter me convidou, mas já disse
que não tenho poderes, nem sei nada especial.

--- Você faz parte desse mundo, é um daqueles que vislumbraram a verdade que
ninguém vê. Também provou ser um homem de valor ao salvar Charlie. Embora não
precisasse intervir, você o fez. Por isso devo agradecê-lo para todo o sempre,
já que, mesmo que chegasse até ele a tempo, não saberia o que fazer.

--- Eu ensinei o truque aos dois, e é apenas isso que sei, acredite.

--- Desculpe por dizer isso, mas duvido. Mesmo que acredite firmemente em sua
própria ignorância, senhor Abercrombie, deve ter amealhado muito mais
conhecimento do que imagina.

Charles voltou à mesa interrompendo a conversa, pois trazia os bolinhos. Todos
pegaram um, e Ian mastigou o seu, pensativo. Aquele convite era com certeza
disparatado, fruto de gratidão ou mesmo de polidez, não fazia sentido
aceitá-lo. Além disso, era imprudente ficar em um lugar só, ainda mais em um
grande centro como Londres, que provavelmente era visitado por seu irmão e seu
pai.

--- Ian? --- disse Victor, cauteloso. --- Sinto por ter que te dizer isso, mas
Charlie me alertou telepaticamente sobre suas reflexões. Seu pai, Scotsman,
faleceu. Eu sinto muito.

--- Como?

--- Não me inteirei das circunstâncias, apenas fiquei sabendo pela imprensa.
Novamente, sinto muito.

Ian parara de mastigar. Engoliu, a muito custo, o que tinha na boca e
levantou-se de supetão indo até o banheiro. Abriu a porta de uma das duas
cabines e vomitou no vaso, ajoelhado, tudo que comera até não restar nada mais
em seu estômago para ser expelido. Levantou-se com dificuldade, lavou o rosto
repetidas vezes e segurou a pia com ambas as mãos, pressionando-a o mais forte
que podia até seus dedos doerem. Olhou-se no espelho. Havia um esgar terrível
na face molhada.

Com o pensamento cristalino apesar de tudo, ele considerou suas opções por um
instante e tomou uma decisão. Seus passos, antes cambaleantes, tornaram-se
firmes e o levaram até os Estranhos de Essex, a quem anunciou:

--- Quero me juntar a vocês. Digam-me o que fazer.
