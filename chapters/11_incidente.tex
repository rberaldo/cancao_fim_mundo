\chapter{O Incidente}

\data{Bacamarte, Minas Gerais\\
Julho de 2004}

O ribeirão Neves parava de correr durantes os meses de seca mais aguda,
deixando em seu curso apenas um fio de água em uma vala de três metros
por dois de profundidade. Cinco pessoas, três policiais locais e dois
forasteiros, estavam lá para investigar a presença de um cadáver que
jazia abaixo de uma pequena ponte de madeira.

Os forasteiros eram José Betto e Kátia Sinatra, agentes da
\textsc{abin}, a Agência Brasileira de Inteligência. Ambos já haviam
estado em locais com ocorrências similares, e aquele era um sinal
ominoso de que algo ameaçador poderia estar por perto.

--- O corpo desse homem foi trucidado por um bicho grande, com certeza.
--- disse o delegado Walter Ferreira.

Ainda era dia, mas a ponte obstruía a luz da área do suposto crime.
Kátia tirou uma lanterna do bolso de sua jaqueta para examinar o
cadáver. O rosto ainda estava intacto apesar de ostentar alguns
arranhões causados pelo “animal”. A agente dirigiu-se a um dos
auxiliares de Ferreira, um jovem que tapava o nariz por causa do cheiro
de putrefação.

--- Sabemos quem ele é?

--- Não, senhora. --- disse o auxiliar, nauseado.

--- Betto, teremos que ligar para o Gil. --- disse Kátia, quase certa do
que tinha de fato ocorrido.

--- Concordo. --- disse Betto. Sua voz era calma, mas olhou para o
relógio com apreensão antes de discar os números em seu celular. ---
Gil? É o Betto. Temos um problema aqui. Vamos precisar de sua presença e
da\ldots\,Prataria. Hum. Ok. Quanto tempo? Ótimo, até mais.

Betto desligou o telefone e, com urgência, disse a Ferreira:

--- Senhor Ferreira, preciso que imponha um toque de recolher para a
noite de hoje. Sem exceções. Todos deverão estar em suas casas quando o
sol se pôr.

--- Está falando sério? O que aconteceu? --- perguntou o homem, perplexo
diante do absurdo daquela situação.

--- Não posso dizer, isso é segredo de estado. Só faça o que eu estou
dizendo. Muitas vidas, e a sua carreira, dependem disso.

--- Olha, seu José, eu sou um funcionário público como o
senhor\ldots\,--- começou a dizer o homem atarracado que, para garantir
que seria levado a sério, começou a se elevar pela ponta dos pés.

Betto percebera a atitude hostil e estava prestes a expor outros
argumentos mais persuasivos quando Kátia Sinatra se interpôs entre os
dois e, de dedo em riste, disse categoricamente para o delegado:

--- Você não é nada como nós! Podemos fazer você ser transferido para a
Amazônia, isso se resolvermos ser bonzinho! Faça o que Betto sugeriu,
agora!

Walter Ferreira não era alguém que podia ser intimidado facilmente e não
queria abrir mão de sua autoridade depois daquele impropério perante
seus auxiliares, mas sabia que estava diante de uma mulher decidida que
tinha poderes advindos de escalões muito mais altos que o seu. Resolveu
aquiescer.

--- Vamos lá, pessoal. --- disse aos dois- Vamos fazer o que os chefes
mandam. --- disse ele, com toda a ironia que ousava exprimir.

Os três deixaram o local, sendo que o jovem auxiliar praticamente
correra quando recebeu a ordem. Betto esperou que eles entrassem no
carro para dizer a Kátia:

--- Você deveria ir com eles, para termos certeza de que tudo correrá
como o esperado. Se houver algum problema, me ligue.

--- Claro. Eu seguirei os três patetas agora mesmo. Gil disse quando
chega?

--- Ele deve estar aqui em poucos minutos. Mandei as coordenadas em uma
mensagem de texto.

--- Até mais, então. --- disse ela antes de se virar e rapidamente ir
até seu carro.

Betto voltou à área em que o corpo estava e se agachou procurando por
mais pistas. Notou um pequeno, quase imperceptível, brilho prateado que
vinha do chão, a alguns metros dali. Levantou-se e foi até ele. Era um
resquício de pó metálico que refletira por alguns instantes a parca luz
solar. Betto apanhou uma caneta de aço inoxidável do bolso de seu paletó
e levou-a até a substância, remexendo-a. Chegou um pouco mais perto e
subitamente uma dor excruciante instalou-se em todo seu corpo, fazendo-o
cair agonizando.

--- Betto!

Gil, um homem baixo e careca que chegara instantes antes, correu até
Betto, que se contorcia. O dedo indicador do agente estava todo coberto
por uma crosta prateada, que se espalhava pelo resto de sua mão.
Discando freneticamente no teclado de seu celular, o recém-chegado
traçava um grande círculo na terra ao redor dele e de Betto, e mal ouviu
a reposta do outro lado da linha quando exclamou:

--- Tio Baca, é o Gil. Meu colega Betto foi envenenado por
\textit{Argentum Toxicattus}. Estou levando-o a sua casa nesse instante.
Por favor, ache um jeito de salvá-lo.

O chão sobre eles cedeu e caíram em um buraco sem fim. A terra
lentamente ascendeu de volta á superfície, mas ambos tinham
desaparecido.

Meia hora depois, na iminência do crepúsculo, Kátia voltou para a ponte
e viu que o outro agente desaparecera. Preocupada, apanhou seu telefone
e ligou para o número dele.

--- Josué vai ficar bem? --- perguntou Kátia, apreensiva.

--- Sim, com certeza. Pode ser que haja algumas sequelas, mas nada muito
preocupante. O mago que está cuidando dele é o melhor que conheço.

--- Ótimo --- disse ela, mais tranquila. --- Quanto ao lobisomem\ldots

--- Kátia, acho que não há lobisomem. Ou pelo menos não há mais.

--- Desculpe, Gilberto, mas estou diante de um corpo cujas entranhas
foram comidas e, a não ser que lobos-guará tenham começado a matar
pessoas de cidades minúsculas, tenho certeza de que foi a porra de um
lobisomem que causou toda essa confusão.

--- Eu vi as mesmas evidências que você, Kátia. Só que acho que há algo
faltando, principalmente porque tem resquícios de \textit{Argentum
Toxicattus} no local. Não tive tempo para nada quando cheguei aí, e sei
que você pode me ajudar enquanto o sol não se pôr. Procure nos arredores
por um sepulcro. Se você o achar, me ligue de volta. Agora preciso ir
ajudar a salvar seu parceiro.

A agente desligou o telefone e guardou-o no bolso de sua calça. Faltava
meia hora para o crepúsculo.

Começou a percorrer todo o perímetro empunhando sua pistola Glock.
Depois de quinze minutos, avistou um arbusto que se erguia de um pedaço
de terra que fora claramente remexida. O sol já começava a se esconder,
mas ela queria ir até o fim. Foi até o carro, abriu o porta-malas e de
lá tirou uma pá que utilizou para cavar um buraco e verificar se as
suspeitas que tinha estavam corretas.

Já estava escuro quando a lâmina da pá encontrou algo mais duro. Kátia
passou a usar suas mãos para tirar cuidadosamente a terra do objeto. Com
sua lanterna, ela verificou que ali jazia mais um cadáver. Este, porém,
estava intacto com a exceção de um de seus braços, que fora extirpado
recentemente pela aparência da lesão.

Kátia ligou novamente para Gil. A não ser que houvesse mais um lobisomem
nos arredores, o caso estava resolvido.

--- Alô, Gil? Como ele está?

--- Está bem melhor. Achamos que não haverá nenhum dano duradouro. Fez o
que pedi?

--- Encontrei o sepulcro. Dentro há um homem cujo braço foi decepado.
Suponho que um de seus colegas tenha feito o serviço sem que você
tivesse conhecimento.

--- Obrigado, Kátia. Esse não é o caso, porém, e sim algo muito mais
preocupante. Deixe o corpo como está, em breve aparecerei aí para colher
algumas amostras.

--- Se esse não é o caso, então o que é?

--- Não posso te dizer. Ainda. Me espere onde encontraram o primeiro
corpo. E fique atenta, em todo caso. Boa sorte.

Kátia colocou o arbusto que fora retirado em cima da vala, de modo a
encobri-la. Voltou ao local e que seu carro estava e entrou para
esperar. Ficara um pouco frustrada com a falta de explicações de Gil,
ainda mais porque aquela era uma situação recorrente que demorava a ser
descontinuada. Tivera sido assim, porém, desde o primeiro dia, e o que
mais se podia esperar de um agente da Ordem de Hermes?
