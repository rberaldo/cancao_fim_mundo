\chapter{O Início}

\data{Franca, São Paulo}

--- Não sei do que está falando. --- disse Ian, surpreso tanto com a revelação
quanto com a naturalidade que aquele rapaz, mais ou menos de sua idade, disse
aquilo. Resolveu responder na mesma moeda: --- Foi meu amigo que saiu daqui
voando, na verdade.

--- Jura? Achei que não fosse ver um Estranho aqui tão cedo\ldots

--- Você sabe sobre os Estranhos?

--- Sim, claro. Desculpe se pareceu desrespeitoso, é esse é o termo usado no
futuro.

--- Futuro?

--- Pois é. Já que estamos trocando confidências apesar de não sabermos nada um
sobre o outro, posso te dizer que essa Kamen aqui é uma máquina do tempo.

--- Então você é um motoqueiro em uma Kamen preta?

--- Se você quer ver as coisas dessa forma\ldots

--- E você viaja no tempo com ela?

--- Foi desse jeito que eu trouxe os dinossauros para cá.

--- Você estava falando sério sobre o pterodáctilo então?

--- Claro! Eu tenho uma Saurológico nas redondezas.

--- Os monstros que eu vi na pintura\ldots\,Eram dinossauros, é claro!

--- Não os chame de monstros\ldots\,Eles não gostam disso. E você não iria
querer ver um Rex puto contigo.

--- Imagino que não.

--- Enfim, de que pintura você está falando?

--- Meu amigo, aquele que voa, pode pintar o futuro\ldots\,E ele pintou dois
caras (parecidos com você e eu na verdade) que tinham um exército de
dinossauros atrás deles. Eles estavam enfrentando um outro homem, que tinha um
bando de lobisomens ou sei lá o que mais atrás deles.

--- Lobisomens e o que mais?

--- Sei lá. Umas criaturas de pés invertidos, cavalos sem cabeça com fogo
saindo de seus pescoços\ldots\,Nunca vi qualquer um desses espécimes
pessoalmente, embora pudesse arriscar dizer que são curupiras e
mulas-sem-cabeça.

--- Tenho recebido algumas informações preocupantes. Acabei de voltar da
floresta amazônica, onde conversei brevemente com um grupo de curupiras. Um
homem tem capturado e escravizado seres fantásticos para construir um exército.
Ele pode ser o cara da pintura, não pode?

--- Talvez. Você tem alguma pista de quem ele pode ser?

--- Não, ainda não. Conte-me, o que mais tinha nessa pintura?

--- Os homens que estavam com o exército de dinossauros tinham um “S” na testa
cada. E tinha umas notas musicais entre os dois grupos\ldots

--- “Não ligue para a cicatriz, não vai doer tanto assim.” Um desses homens sou
eu, no futuro!

--- Como você sabe?

--- Meu “Eu” do futuro veio até mim uns meses atrás e me alertou
sobre\ldots\,Algumas coisas. E ele tinha uma cicatriz na testa, justamente em
forma de “S”!

--- Ele falou algo sobre se aliar a outra pessoa?

--- Não. Você pode ser esse outro homem, vai saber. Mas o que me preocupa agora
é descobrir quem é o homem que tem violado a Diretriz Monteiro em benefício
próprio, e o que diabos ele planeja.

--- Diretriz Monteiro?

--- “Não interferir em assuntos sobrenaturais de criaturas assim-chamadas
mágicas sem permissão”. Esse cara ultrapassou essa linha com méritos, e vai ter
que pagar por isso. Vamos lá, suba aqui. Vamos ter uma conversa com alguém que
pode nos ajudar.

\data{Barretos, São Paulo}

Completamente relaxado, Alden Abercrombie estava sentado de frente a um homem
nervoso. Em cima da mesa que os separava, dois copos e uma garrafa de uísque.

--- Vamos, Bonasera, pode beber.

--- Prefiro não, Abercrombie.

--- Ora, deixe disso. Nossa conversa não precisa ser de todo desagradável.

Alden levantou-se, apanhou seu copo, e foi até uma pequena mesa, onde havia
outras garrafas do uísque Abercrombie e um balde de prata com gelo. De lá tirou
dois cubos e inseriu-os no copo. Voltou à mesa, sentou-se e retirou a rolha da
garrafa, vertendo três dedos de uísque em seu copo. Bonasera tremia em um misto
de medo e raiva diante da arrogante calma de seu anfitrião.

--- Então, --- disse Alden, balançando o copo em movimentos circulares ---
vamos ao que interessa. Você está aqui por causa de sua filha e, julgando seu
nervosismo e ira mal contidos, sabe como a coisa se desenrolará.

--- Mia está doente, e nenhum médico ou mago consegue curá-la. Vim até aqui
pedir, como um favor pessoal, que dê um jeito nisso.

--- E como sabe que eu, entre tantos outros magos talentosos, da Ordem de
Hermes ou não, posso ajudá-lo?

--- Porque\ldots\,Porque\ldots\,--- balbuciou Bonasera, enxugando o suor de sua
testa.

--- Porque fui eu que a infectei em primeiro lugar. Você sabe disso, não?

Bonasera calou-se e segurou o tecido de sua calça com força. Olhando para
baixo, incapaz de encarar o sorriso maligno de Alden, assentiu com a cabeça.

--- Você está certo em ficar com medo, Bonasera. --- disse Alden subitamente
sério e ameaçador. --- Está muito certo. Poderia matá-lo aqui mesmo, sem ao
menos me esforçar. Sou, no entanto, um homem extremamente caridoso e vou
deixá-lo me fazer um favor em troca da saúde de sua filha.

Bonasera continuou evitando o olhar de Alden, que bebericou seu uísque pela
primeira vez.

--- Sua rentável indústria de cosméticos será transferida para o meu nome. Você
continuará sendo o diretor de jure, mas acatará todas as decisões que eu tomar
em troca do mesmo pro labore que você recebe atualmente, o que não é pouco
dinheiro. Afinal de contas, um homem precisa prover para sua família, e eu
jamais tiraria isso de nínguem. Acha isso razoável?

Relutantemente, Bonasera olhou para ele e assentiu com a cabeça.

--- Os documentos, como deve saber, já foram preparados. Você só precisa ---
disse, enquanto abria uma gaveta --- assiná-los.

À sua frente, Bonasera viu um contrato de transferência de posse já assinado
por um juiz, duas testemunhas e pelo próprio Alden. Havia apenas uma linha em
branco em cada uma das cópias do documento.

--- Eu assinarei, claro. Só quero que você cure minha filha antes.

Alden juntou em formato de concha os dedos de sua mão esquerda, a que não
segurava o copo, e a girou, vendo Bonasera silenciosamente contorcer-se de dor
em sua cadeira.

--- Não sei porque essa falta de fé em mim, Bonasera.

Agonizante, incapaz de produzir um som sequer por conta do intrincado feitiço
lançado por Alden, Bonasera lançou-lhe um olhar clamando por piedade, tal era
sua dor.

Alden relaxou os dedos vagarosamente enquanto seu convidado se curvava para
frente, respirando fundo e contraindo seus músculos dos braços e das pernas. No
ínterim, Alden abriu uma segunda gaveta e colocou na mesa, ao lado dos papéis,
um frasco de vidro esverdeado.

--- Dê isso para Mia. Faça-a beber todo o conteúdo, de uma vez. Em algumas
horas os sintomas desaparecerão. --- disse ele.

Bonasera pegou o vidro e colocou-o no bolso interno do paletó. De outro bolso,
externo, tirou uma caneta e assinou nos lugares em que seu nome estava
impresso.

--- Pronto, Sr. Abercrombie.

--- Ótimo, Bonasera. Ótimo. Fico feliz que tenha sido tão solícito. --- disse
ele em um tom mais jovial, pegando os papéis e colocando-os de volta na gaveta.
--- Cuide bem de sua filha e descanse. Segunda-feira você já terá de fazer
algumas mudanças, e isso exigirá muito trabalho.

--- Quais\ldots\,Quais mudanças, senhor?

--- Ah, coisas mínimas. Em primeiro lugar, expandiremos um pouco a venda dos
seus produtos ou, melhor dizendo, nossos produtos.

Mal contendo seu ressentimento, Bonasera indagou:

--- Expandir para onde? Dominamos o mercado interno e já exportamos para a
Argentina e para o Paraguai.

--- Sim, e isso é excelente, claro, mas não o suficiente para a qualidade
oferecida. Poderíamos reinar em outros lugares, e é o que vamos fazer. A
começar pelos EUA.

--- O senhor deve saber que não podemos fazer isso.

--- Por que? --- perguntou Alden, sorrindo.

--- Por causa da Svenson. Eles não permitem competição nesse nicho do mercado
em terras norte-americanas. E o senhor deve saber o que eles fazem com aqueles
que os desrespeitam.

--- Ah, isso. --- disse, despreocupado. --- Fique tranquilo, Bonasera. Eu me
preocupo com a Svenson. Apenas aumente a produção em 150\% e minha empresa de
marketing elaborará uma campanha para a inserção de seus --- ou melhor, nossos,
preciso me acostumar --- produtos no mercado norte-americano.

--- E o capital?

--- Investirei o dinheiro, pode ficar tranquilo. Cortes nos salários dos
funcionários também serão efetivos para nossos propósitos. O que acha de uma
redução de 10\%?

--- Os funcionários ficarão furiosos, senhor.

--- Eles não têm muita alternativa, têm? São apenas massa de manobra.

--- Sim, senhor.

--- Ótimo, estamos entendidos a esse respeito, então. A segunda coisa que
deverá fazer é criar um departamento de desenvolvimento de produtos
experimentais.

--- Não entendo, senhor.

--- Armas, Bonasera. Quero desenvolver armas químicas com técnicas mágicas.

--- Novamente, senhor, tenho que avisá-lo que\ldots

--- Eu já disse, Bonasera, --- disse em um tom frio --- eu me preocupo com
isso. A única coisa que você deve temer é sua deslealdade àquele que está
cuidando de você: eu.

O homem assentiu novamente. Estava preocupado com as ordens que Alden lhe dera,
mas estava ansioso demais em ver sua filha recuperada para protestar.
Levantou-se ao notar que estava dispensado e andou a passos rápidos em direção
a porta, saindo com alívio daquela atmosfera aterrorizante.

Alden ponderou por alguns instantes sobre o futuro bebendo o resto de seu
uísque. Colocou o copo na mesa e se serviu de outra dose, encobrindo de líquido
âmbar os cubos de gelo que se dissolviam em água.

Uma batida na porta tirou-o de seu transe e ele pediu para que a pessoa
entrasse. Era seu assistente, Tolo.

--- Senhor, tenho uma notícia.

--- O quê?

--- Seu irmão está novamente no país, senhor Abercrombie. Mais precisamente em
Franca, aqui perto.

Alden congelou. Não via seu irmão havia anos, e por mais que o tivesse
procurado, não conseguira encontrá-lo. Até aquele momento. Levantou-se,
ansioso, depositou o copo na mesa em que estava o balde de gelo e foi até o
mancebo em que seu paletó estava pendurado.

--- Excelente, excelente. Vamos fazer lhe fazer uma visita. Apenas faça-me um
favor antes, Tolo. Traga de minha adega especial cinco daquelas garrafas com
tarjas vermelhas para cá o mais rápido possível. Vou precisar delas para falar
“oi” mais apropriadamente.
