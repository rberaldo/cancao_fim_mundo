\chapter{O Salto de Fé}

\data{Franca, São Paulo\\
Quarto de hóspedes de Tio Baca}

--- Escapamos de uma bela enrascada, hein? --- disse Érico.

--- Mas entramos em outra bem maior. Não sei se quero fazer isso, entende? ---
disse Ian Abercrombie.

--- O que? Lutar contra seu irmão?

--- Exatamente. Eu saí do Brasil para nunca mais ter que me envolver nos
negócios da família, e agora estou enfiado até o pescoço neles.

--- Pelo que entendi, você não quis compactuar com o jeito deles de fazer
negócio. Quer forma melhor de manter seus princípios?

--- Minha principal meta era nunca mais ter que sequer ouvir falar deles. Em
minhas primeiras horas aqui, não só fui torturado por um capataz com nome
idiota do meu irmão como também me envolvi em uma conspiração que envolve
sacis, curupiras e sei lá mais quais criaturas folclóricas. Você realmente acha
que estou satisfeito, ou mesmo disposto, a fazer isso?

--- Ian, eu não sei se o que estamos fazendo é uma boa ideia ou não, mas sei de
uma coisa: esse é um trabalho de duas pessoas, pelo menos. Eu não posso fazer
isso sozinho, não quero.

--- Mas você terá outros ao seu lado, como o Tio Baca, ou o tal Joca
Betto\ldots

--- José, na verdade.

--- Enfim, eu sei que a pintura diz que estamos destinados a lutar contra meu
irmão, mas eu não acho que meu destino está completamente selado.

--- Não é apenas por causa da pintura que você precisa fazer isso. Você não
sente algo dentro de você, algo que diz que podemos fazer a diferença?

--- Não, não de verdade.

--- Então o que você está fazendo aqui, Ian? Por que você voltou? Por que você
acha tão difícil acreditar?

--- Por que você acha tão fácil?

--- Nunca foi fácil! --- disse Érico, emocionado. --- Isso é um salto de fé,
Ian.

Houve um silêncio desconfortável, que perdurou alguns minutos. Mesmo relutante,
Ian acenou positivamente com a cabeça.
