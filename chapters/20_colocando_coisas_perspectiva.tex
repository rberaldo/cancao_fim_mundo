\chapter{Colocando as Coisas em Perspectiva}

A ilha de Jira, como seria conhecida a partir do século~\textsc{xix}, data de
seu descobrimento, fora escolhida por Érico para abrigar seu Saurológico em um
ponto do tempo que não interferisse com a evolução da humanidade. Levara para
lá diversos espécimes a fim de estudá-los e principalmente para apreciar sua
beleza majestosa. Ainda assim, Érico começou a sentir novamente um vazio, uma
falta de propósito.

Depois de poucas semanas, alguns dinossauros começaram a adoecer por falta de
vegetação apropriada, outros escaparam. Érico levou os que não se adaptaram de
volta, capturou os fugitivos e manteve apenas algumas espécies lá.

O peso de seu parcial fracasso precipitou Érico a imaginar outros jeitos de
usar a Kamen. Queria ajudar, salvar pessoas, o mundo. Quando soube pela
televisão que em São Paulo uma menina de 12 anos era mantida refém em seu
apartamento, ameaçada por um ladrão que lá procurara abrigo, ele decidiu entrar
em ação. Calçou seu recém-comprado coturno preto, vestiu a camisa xadrez que,
sabia, manteria por anos à fio, e se teleportou para as mediações da casa da
menina.

A Kamen sempre aparecia em lugares que não estavam expostos a olhos humanos, o
que no caso significava estar a uma distância razoável da área ocupada pela
polícia e pela mídia. Érico parou o tempo e dirigiu até o ponto mais acessível.

O prédio, parte de um condomínio popular, não tinha portaria propriamente dita,
apenas um portão escancarado, guardado naquele instante por um policial.
Repórteres e \textit{cameramen} se apinhavam atrás do cordão de segurança que
ficava a meros centímetros do portão. Érico teve que se esgueirar pelos corpos
imóveis para adentrar, subir as escadas também cheias de policiais e chegar até
a porta do apartamento. Uma vez ali, parou. Não sabia como arrombar uma porta,
só as vira sendo arrombadas em filmes. Resolveu copiar: afastou-se um pouco e
jogou-se de ombro direito contra ela, caindo no chão sem sucesso. Levantou-se,
massageando o ombro dolorido e xingando. Resolveu chutar até que a fechadura
cedesse. Apenas do vigésimo terceiro chute a porta foi aberta e ele, mancando,
entrou no apartamento.

O ladrão estava escondido ao lado da janela, mostrando em direção à rua a
menina e o cano do revólver que apontava para sua cabeça. Érico foi até o
homem, puxou o braço esquerdo que agarrava a menina para longe, tirou-lhe a
arma e levou a menina para a parte de fora do prédio, entre o portão e a porta
de acesso aos apartamentos. Voltou à casa e pegou o revólver. Olhou longamente
para o ladrão e em seguida socou seu rosto com toda a força que podia.

Ainda mancando, com a mão direita imobilizada pela dor e o ombro latejante,
Érico lentamente caminhou até a Kamen com a melhor sensação do mundo.

\data{Dois dias depois}

Érico estava em uma lanchonete com um laptop aberto em sua frente, procurando
na internet casos em que pudesse ajudar assim que tirasse o gesso da mão e se
livrasse da bota ortopédica.

Olá.

Érico olhou acima da tela. Um homem negro, com um terno bem cortado e um
sorriso agradável estava sentado à sua frente, sem que ele tivesse percebido
sua chegada.

Meu nome é José Betto. Betto com dois “t”s. --- disse ele, estendendo a mão
esquerda para que que a apertasse. Relutantemente, Érico o fez, sem seguida
perguntando:

--- Posso ajudar?

--- Na verdade sim. Não com o amato quebrado, ou com a luxação no pé, mas pode.

--- Você sabe o o osso que quebrei.

--- É meu trabalho saber. Na verdade, meu trabalho é encontrar coisas estranhas
pelo Brasil e privar o público de seu conhecimento. Sou um agente da
\textsc{abin}, e trabalho na divisão paranormal, ou, como costumam chamar
jocosamente, no Arquivo~X. Esse é meu emprego; o que faço de verdade é muito
melhor.

Intrigado, Érico resolveu entrar no jogo do estranho homem.

--- O que faz de verdade?

--- Eu ajudo a salvar a mundo.

--- Certo\ldots

--- Por que a surpresa? Você fez isso há dois dias. Quer dizer, você deu uma
pequena contribuição para isso, a um alto custo.

--- Como diabos você\ldots?- disse Érico, exaltado.

--- Eu disse que meu trabalho é saber de coisas. Mais especificamente, as
imagens de uma câmera de alta resolução a que tive acesso revelaram seu
movimento em direção à casa da menina. Não se preocupe, as imagens já foram
“perdidas”, embora não houvesse meio de o reconhecer mesmo com elas. Consegui
chegar até você pois tenho meios para fazê-lo.

--- Que meios?

--- Isso vem depois. Primeiro você tem que passar por uma entrevista.

--- Entrevista?

--- Sim, uma entrevista de emprego.

--- Para fazer o que?

--- Salvar o mundo, oras.

Desculpe, você parece ser um homem sério, mas isso a cada momento soa mais como
uma piada de péssimo gosto.

Jura? Eu acho natural. Houve uma época em que ouvir isso me faria rir, quando
eu achava que minha transferência para a divisão paranormal era a maior punição
possível. Desde então eu quase fui morto por lobisomens, enfrentei um mago
fora-da-lei e consegui prendê-lo, mesmo sabendo fazer apenas magia rudimentar e
já salvei diversas pessoas. A ideia de salvar o mundo, e lidar com o
sobrenatural, se é que podemos chamá-lo disso, tornou-se fácil, embora a
prática seja sempre complicada. Você tem vinte anos e consegue parar o tempo!
Como pode achar isso tudo uma piada?

Érico considerou o que o homem dizia.

--- Lobisomens?

--- Sim, eles existem.

--- Magia?

--- Sei uma coisa ou outra. Por exemplo:

Betto pegou o saleiro, desenroscou a tampa e deixou cair todo o conteúdo em
cima da mesa. Postou o indicador, ereto e em posição verical, da mão esquerda
ao lado.

--- Digamos que esse seja eu. --- disse, referindo-se ao indicador. --- E eu
preciso me defender de fantasmas bravos. Não tempo para manualmente fazer um
círculo que me proteja, então despejo o sal e\ldots

Com o indicador direito, tocou brevemente os grãos e fez um movimento circular
fluido. O sal imediatamente se dispôs em um círculo perfeito ao redor do
indicador esquerdo.

--- Voilá. Proteção garantida.

--- Uau.

--- Eu sei, isso é muito legal. Quem me ensinou isso pode vir a ser seu futuro
empregador.

--- O que eu teria de fazer, exatamente?

--- Primeiro vem a entrevista. Fizemos nosso dever de casa, sabemos todo seu
histórico, inclusive as coisas embaraçosas. Não tem problema --- disse, ao
notar o constrangimento de Érico --- compreendemos o que aconteceu. Você está
aqui hoje, de pé, e é isso que interessa. O que será avaliado em seguida é a
sua disposição e algumas outras coisas que só ele conseguirá averiguar.

--- “Ele” é o empregador?

--- Sim, Tio Baca.

--- Tio\ldots\,Baca\ldots?

--- Exato. É ele quem colocará as coisas em perspectiva.
