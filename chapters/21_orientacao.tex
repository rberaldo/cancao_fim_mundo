\chapter{Orientação}

\data{Franca, São Paulo\\
 Casa de Tio Baca}

--- É melhor acordar, meninos.

--- O que aconteceu? --- perguntou Ian, assustado com seu súbito
despertar.

Tio Baca estava com o jornal \textit{Comércio da Franca} dobrado em uma
de suas mãos. Ele abriu-o e leu o que parecia ser a principal manchete
do dia:

--- “Jovens meliantes matam homem em Bacamarte.” Está neste e em todos
os jornais da região.

Com certa irritação (ele detestava acordar cedo), Ian sentou-se na cama
e olhou para a edição que Tio Baca mostrava.

--- E o que isso tem a ver com\ldots\,Porra!

Ian Abercrombie viu seu rosto e o de Érico Porto estampados na primeira
página.

--- Como isso aconteceu? --- perguntou Érico, igualmente espantado.

--- Não sei. Mas suspeito que exista envolvimento do outro irmão
Abercrombie. --- disse Tio Baca calmamente. Ele parecia querer evitar
maiores exaltações.

--- Alden deve ter armado para nós. Ele com certeza pode fazer isso. ---
disse Ian.

--- Aqui diz que a polícia tem um vídeo provando nosso envolvimento. ---
Érico pegara o jornal para ler a matéria por completo.

--- Outra artimanha dele. Droga! Eu sabia que voltar ao Brasil havia
sido uma má ideia.

--- Não tem apenas um vídeo, mas também digitais, \textsc{dna},
tudo\ldots\,Ele fez um ótimo trabalho. --- disse Érico, dobrando o
jornal após ter desistido de ler o resto. Ele viu as palavras “ritual
satânico” tentando explicar a razão do homicídio, e não teve estômago
para continuar.

--- E agora? Nosso plano de combatê-lo foi por água abaixo antes mesmo
de começarmos. --- perguntou Ian.

--- Vocês devem continuar seguindo o plano. Alden quer exatamente isso:
assustá-los até o ponto de fazerem uma besteira ou desistirem.

--- Acho que uma grande besteira nesse ponto seria não desistir. ---
disse Ian, tentando fazer Tio Baca ouvir a voz da razão.

--- E para onde você iria, Ian? Vai voltar para a Europa ou dessa vez
irá para um lugar ainda mais distante? --- disse Tio Baca.

--- \ldots

--- Fugir não é a solução. Nunca foi. O que vocês precisam fazer é
seguir minhas recomendações. Érico, a essa altura Betto já deve estar em
seu encalço para esclarecer o caso. Você deve explicar tudo para ele,
nos mínimos detalhes. Ele irá acreditar, e ficará do seu lado. Já
alertei Gil e Jorge. Eles estão a caminho.

--- Devo partir agora? --- perguntou Érico.

--- Imediatamente. --- Érico calçou os sapatos e saiu do quarto em que
estavam dormindo sem pestanejar. --- Ian, você fica aqui. Você precisa
cuidar de suas feridas. E assistir a um vídeo.

--- O que? --- perguntou ele, ainda mais perplexo.

--- Você precisa de orientação, Ian. O seu treinamento com Peter foi
apenas o primeiro estágio de seu aprendizado.

\data{Bacamarte, São Paulo\\
 Chefatura de Polícia}

Entrevista coletiva da agente da especial da \textsc{abin} Kátia
Sinatra.

--- Srta. Sinatra, você pode nos dar uma declaração sobre o caso
Abercrombie/Porto?

--- Bom dia, senhoras e senhores. Há quatro dias, um corpo foi
encontrado dilacerado aqui na cidade de Bacamarte. As evidências
sugeriram que a vítima havia sido atacada por um animal feroz. Vinte
quatro horas depois, outro corpo foi achado, dessa vez com um braço
decepado. Esse modus operandi, apesar de aparentemente incomum, foi
relatado em 23 outras cidades do Brasil nos últimos quatro anos. Duas ou
três vítimas brutalmente assassinadas em um intervalo de tempo de doze a
24 horas, seguidas por um outro corpo achado com um dos braços
decepados.

“Acreditamos que esses homicídios façam parte de rituais satânicos
promovidos por Érico De Almeida Porto e Ian Miguel Abercrombie, dois
jovens desajustados que haviam desaparecido há certo tempo. Ian, irmão
mais novo do industrial Alden Abercrombie, sumira há dois anos, logo
depois de seu décimo-oitavo aniversário. Já Porto morou em várias
cidades, sem nunca ter uma ocupação ou moradia fixa, até sumir
completamente da face da Terra há cinco meses. Os desaparecimentos dos
dois nos períodos dos assassinatos anteriores, aliados às provas
circunstanciais que recolhemos nesses novos crimes nos fazem concluir,
portanto, que eles são os perpetradores dessa série de crimes hediondos.

“A \textsc{abin} está, a partir desse momento, assumindo as
investigações do caso Abercrombie/Porto. Nós não pouparemos esforços
para esclarecer minuciosamente o que aconteceu, e capturá-los para que a
justiça seja feita. Isso é tudo por hoje, obrigada.

\data{Franca, São Paulo\\
Casa de Tio Baca}

Ian estava sentado em frente à televisão na sala de estar, boquiaberto
com o que acabara de ver.

--- Sou procurado por mais de cinquenta assassinatos!

--- Não se apoquente com isso, não ainda, não agora. Você precisa ver
isso aqui.

--- O que é isso?

Tio Baca colocou uma fita no seu quase pré-histórico vídeo cassete, e
apertou play. Na tela, surgiu um letreiro em preto-e-branco que dizia
“orientação”.

Em seguida, um homem asiático surgiu vestindo um jaleco branco. Atrás
dele, apenas um fundo igualmente branco.

--- Olá. Eu sou o Doutor Marvin Kolleritz. Este é o primeiro vídeo de
orientação que você, Ian Miguel Abercrombie, irá assistir para iniciar
seu treinamento nas artes de salvar o mundo.

--- Como ele\ldots

--- Shh! --- disse Tio Baca, extremamente concentrado.

--- Primeiramente, devo dizer que esse vídeo se adapta a cada indivíduo
que o assiste. Por isso, não estranhe, isso não é um engodo; é apenas a
programação intrínseca dele que está sendo seguida.

Mas vamos ao que interessa. Para salvar o mundo, você precisa entendê-lo
e apreciá-lo. Essa é uma tarefa extremamente árdua, pois poucas pessoas
se propõem a seguir o caminho que leva a esse nível de esclarecimento. A
maior parte prefere deixar-se consumir pela entropia, aguardando o fim
de sua existência insatisfatória. Eles não percebem que fazem parte do
mundo. Eles não percebem que o constroem a cada ação, por menor que
seja. Eles não percebem que podem salvá-lo se quiserem. Você, Ian,
passou tempo demais escondido, fugindo dessas responsabilidades. Se
decidir continuar assim, pode apertar stop; nada acontecerá e você
voltará a sua vida normal. Vou dar um tempo para que possa pensar.

Ian apenas olhou fixamente a tela, indeciso. Por fim, lembrou-se da
conversa que tivera na noite anterior com Érico, e decidiu dar seu salto
de fé.

--- Continue.

--- Muito bem. Sua primeira lição é: os universos que compõem o
Multiverso não são paralelos. Eles convergem, todos os eles, em um
ponto, apenas um. E esse ponto pode ser acessado por meio dos sonhos.
Lembre-se disso.

Isso é tudo por hoje. Namasté. E boa sorte.

A tela ficou toda preta, subitamente. Ian olhou para o lado, e viu Tio
Baca acompanhado por dois outros homens. Ambos eram baixos, mas um deles
era careca enquanto o outro tinha cabelos pretos muito volumosos.

--- Ian, esses são Gil --- disse Tio Baca, apontando para o que era
careca. --- e Jorge. Eles estão aqui para acompanhá-lo.

--- Acompanhar-me para onde?

--- Para o mundo dos sonhos. --- disse Gil. --- Você precisa tornar-se
um onironauta antes de continuar sua jornada.

Jorge colocou sua mão direita dentro de seu emaranhado de cabelos e de
lá tirou uma caixa de madeira.

--- Como\ldots\,---perguntou Ian, intrigado com o estranho fenômeno.

--- Shh\ldots\,--- disse Jorge, tirando um pó metálico da caixa e
soprando-o na direção de Ian. --- Durma. Sonhe. Nos veremos do outro
lado do espelho.
