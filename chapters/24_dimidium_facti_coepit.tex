\chapter{Dimidium Facti Qui Coepit}

\data{Franca, São Paulo\\
 Casa de José Betto}

--- Você tem ideia da besteira que está fazendo ao vir aqui? --- disse José Betto ao abrir a porta para Érico Porto. --- de agente de ligação entre você e a \textsc{abin}, eu agora sou suspeito de ser seu cúmplice em uma série de crimes absurdos.

--- Não se preocupe, José. Eu parei o tempo. Ninguém saberá que estive aqui. --- disse Érico, com problemas mais sérios em mente. --- Mas você deve imaginar que eu não fiz nenhuma dessas coisas.

--- Eu sei que não, mas meus superiores e a opinião pública acham que sim. E quem é essa porra de Ian, afinal?

--- Um cara que conheci uns dias atrás. O irmão dele nos sequestrou, e agora armou para nós. É ele que anda capturando criaturas mágicas e formando um exército.

--- Então Tio Baca não estava delirando? Kátia acobertou tudo?

--- Aparentemente. Enfim, lidaremos com ela depois. Tenho que descobrir tudo que puder sobre Alden Abercrombie.

Houve uma pausa, em que Betto pereceu meditar profundamente. Ele conhecia aquele nome, e não por causa da seção de fofocas de revistas que se intrometem na vida dos ricos e famosos.

--- Ok. --- disse ele, com extrema relutância. --- Acho que é hora de procurar Xavier Constantino.


\data{São Paulo, São Paulo}

Érico e Betto teleportaram-se à Tabacaria do Tom, uma zona neutra para todos os tipos de negócios e criaturas.

--- Kamen, reverta a pausa temporal para este cavalheiro de cabelos extremamente bem penteados. --- disse Érico, observando que, mesmo em uma vizinhança barra-pesada, os cabelos de Xavier Constantino estavam impecáveis.

--- O que\ldots --- disse ele, momentaneamente baqueado pelo efeito cronal.

--- Ele pode parar o tempo, Xavier. Você já viu coisas mais estranhas que isso. --- disse Betto, sem muita paciência. Ele explicara para Érico que Xavier Constantino era um traficante de informação, o que significava que ele sabia de todos os acontecimentos do mundo sobrenatural. Portanto, não se surpreenderia com as palavras “curupiras”, “sacis” e “exercito sobrenatural” em uma mesma frase. --- Enfim, esse é meu amigo, Érico Porto.

Já recobrado, Xavier assumiu uma postura empertigada, quase arrogante. Aproveitou a oportunidade para dar uma boa olhada no decote de uma das atendentes da tabacaria antes de dizer:

--- O viajante temporal. Já ouvi os rumores. Mas o fato de você ser um caçador de lobisomens é novidade para mim.

--- Eu não sou um caçador. E como diabos você sabe disso?

--- Qualquer criança pode perceber a relação entre cadáveres dilacerados, homens mortos com membros extirpados e as datas dos assassinatos. O problema é que a Ordem de Hermes, que deveria ser responsável por não deixar essas mortes acontecerem, há muito tempo tem feito vista grossa.

--- Para que?

--- Eu ganho a vida vendendo informações, acha que vou te contar estorinhas por diversão? Você é alguma espécie de criança que não entende o conceito de pagamento? --- perguntou ele com desdém. Virou-se para Betto apontando para Érico. --- José, por que você traz esses imbecis a tiracolo?

--- Não há necessidade para esse tom, Xavier. Érico é meu convidado, e você irá responder cada pergunta que ele fizer, ou sofrer as consequências. Ambos sabemos que qualquer informação que você nos der vale menos que sua liberdade.

--- Não qualquer uma. Algumas podem valer meu pescoço. Mas o mero fato de vocês estarem aqui sem nenhuma interferência já prova que eu posso falar sobre o assunto. Isso, José, é sobre a Svenson.

--- Como isso pode ser sobre a Svenson?

--- Muito simples. Como você já sabe, eles ditam os rumos da Humanidade há séculos\ldots

--- O que? --- perguntou Érico, interrompendo Xavier.

--- Seu garoto é burro e mal-educado.

--- Ele acha que a Svenson é a fachada para uma organização que controla o mundo secretamente. --- disse Betto, já extremamente impaciente.

--- Não acho, eu sei.

--- Mas a Svenson fabrica roupas, não é?

--- A Svenson existe desde os tempos da Revolução Industrial. Caso você não tenha estudado Historia, os primeiros bens manufaturados foram tecidos, a matéria-prima para roupas. Desde então, essa tem sido a fachada deles. Mas o grupo tem inúmeros outros empreendimentos hoje em dia: refinarias de petróleo, redes de televisão ao redor do mundo, jornais, tudo o que você puder imaginar. Eles controlam praticamente todas as fontes de informação do planeta, têm mais dinheiro que deus e têm em suas fileiras todos os tipos de soldados para sua proteção, além de possuírem a pedra filosofal, o que conferiu imortalidade a eles. Os sete sócios são os verdadeiros donos do mundo, e se você não quer aceitar isso, paciência.

--- Boa Teoria de Conspiração, Xavier. Não posso negar que eles são poderosos, mas o que diabos isso tem a ver com o que estávamos falando? --- disse Betto irritado.

--- A Ordem de Hermes, com a criação da Diretriz Monteiro, passou a não interferir em assuntos sobrenaturais a não ser em casos de extrema necessidade. Você, garoto do tempo, já deve ter ouvido a estória. O que isso causou também é notório: a boa e velha recessão. Muitos magos, pouco a fazer, menos dinheiro circulando. O submundo ganhou poder, as atividades ilícitas aumentaram vertiginosamente. Isso deveria equilibrar as coisas, dando o que fazer aos bons companheiros da Ordem, mas ninguém mais se importava àquela altura. E é aí que entra o benfeitor Alden Abercrombie. Ele dá uma pequena contribuição para alguns deles, fica sabendo de algumas informações privilegiadas, como a localização de um lobisomem ou dois, e todos fingem que nada aconteceu. Seria tudo muito bonito, o menino rico com seu exército de aberrações, mas ninguém da Svenson deixaria isso acontecer. Se eles controlam tudo, por que iriam querer alguém amealhando poder? Bem, a resposta para isso é que Abercrombie tem alguma coisa a seu favor, algo que impede a Svenson de simplesmente obliterá-lo.

--- E o que seria isso?

--- Você acha que eu estou fazendo suspense? Isso não é um programa de auditório. Eu não sei. Deve ser algo assustador, com certeza, e acho que não quero descobrir.

--- Então a visita que você recebeu de Natália Sampaio foi a nome da Svenson? Foi ela que pediu o dossiê sobre Alden, não foi?

--- Um bom espião, como sempre, José. Natália e eu nos conhecemos há bastante tempo, e até hoje ela não aprendeu a cobrir seus rastros. Sim, ela encomendou um dossiê, provavelmente em nome da Svenson. Eles têm outros métodos de obter informação, mas imagino que minha perspectiva tenha interessado a eles. Senti-me um pouco lisonjeado, na verdade.

--- Bem, muito obrigado pela informação. Gostaria, é claro, de uma cópia desse dossiê. --- disse Betto, já pronto para ir embora.

--- Tudo o que contei a vocês está lá. Mas o que escrevi foi apenas uma fração do todo. O dossiê de verdade, com informações que eu não poderia acessar, está na sucursal da Svenson em Capadohcia. Pelo que ouvi, é lá que eles estão agrupados para lidar com a situação.

--- Então é para lá que iremos. --- disse Érico.

--- O lugar é uma fortaleza. Vocês irão precisar de ajuda, mesmo com o agente secreto e a motoca do tempo. Procurem o Contista, em Capadócia mesmo. Conhecem?

--- Sim. --- disse Érico, deixando transparecer seu leve rancor pelo homem que conhecera em uma malsucedida missão.

--- Você causa uma impressão daquelas, hein, garoto do tempo? --- disse Xavier.

Érico sentiu vontade de mandá-lo fazer algo bem desagradável, mas conteve-se. Montou na Kamen e esperou Betto subir na garupa antes de ligar a moto.

--- Você disse algo anteriormente sobre estarmos aqui sem nenhuma interferência, e como isso te permitia falar sobre o assunto. O que você quis dizer? --- perguntou Érico.

--- José é um espião competentíssimo, eu sou o traficante de informações especiais mais notório nessa cidade e você tem uma moto do futuro. Você realmente acha que ninguém está observando e que, se não fosse para receberem a informação que eu acabei de dar para vocês, todos nós já não estaríamos mortos? Existem coisas maiores e mais poderosas que a sua Kamen. Não perca isso de vista, garoto.

