\chapter{The Man With The Plan}

\data{Capadócia, São Paulo\\
Sucursal da Svenson}

Érico Porto estava amarrado a uma cadeira, desacordado e com uma
cicatriz em forma de “S” na testa. Um equipamento era retirado de sua
frente por dois dos homens da força-tarefa destinada a combater Alden
Abercrombie enquanto Natália Sampaio, que tivera uma rápida reunião com
um emissário da Matriz, estava sendo posta a par do ocorrido em sua
ausência.

--- Inscrevemos o símbolo da companhia na testa de Porto com o laser,
sem anestesia, como foi pedido. Calculamos que ele acordará em alguns
minutos. --- disse Renata Gomes, a porta-voz da missão.

Pouco depois, Érico acordou, urrando. O Engenheiro alertara-o para a dor
que sentiria, mas não sabia que seria tão intensa e profunda. Não era
apenas a testa que o afligia; sua cabeça inteira parecia prestes a
explodir.

--- Ah, Sr. Porto, isso são modos. --- zombou Natália. --- Um homem de
brios não grita como uma criancinha por causa de um machucado desses...

--- Sua cadela, para que porra de razão você fez isso? --- gritou Érico,
enquanto lágrimas caíam involuntariamente de seus olhos.

--- Você é como gado; alheio a tudo, estúpido, ignorante. Merece ser
marcado como tal. --- disse ela, rindo. --- Agora você é propriedade da
Svenson.

--- Você acha, sua puta desvairada, que uma merda de marca vai me fazer
trabalhar para vocês? Foda-se!

--- Guarde suas ofensas para quem se importa, Porto. O corte na cabeça é
apenas simbólico, tivemos a ideia ao ler seu diário de merda. Nossa
garantia de que irá fazer aquilo que mandarmos é o dispositivo alojado
na parede de seu crânio. É por isso que sua cabecinha dói tanto...

Érico percebeu que, apesar da dor retumbar por toda sua cabeça, os dois
focos estavam na testa e em um ponto acima de sua nuca.

--- O chip dá sua localização precisa, e contém um quinto de mililitro
de nitroglicerina. Se tentar retirá-lo, ele explode. Se não fizer
exatamente o que mandarmos, quando mandarmos, ele explode. Se você for
para outro lugar que não seja Jira nas próximas vinte quatro horas, ele
explode. É muito simples, na verdade. --- disse Natália, triunfante. ---
Renata, diga ao nosso garoto o que ele precisa fazer.

Renata Gomes girou a cadeira em que Érico estava amarrado para que ele
visse o monitor principal, que mostrava um desenho de uma caixa
retangular de madeira fechada por um cadeado.

--- Aquele é o Receptáculo. Sua missão é obtê-lo, evitando a qualquer
custo que ele seja aberto. Se isso acontecer\ldots\,--- disse ela,
hesitante. --- \ldots\,você será morto no ato.

Renata prosseguiu.

--- Sabemos que há pelo menos um homem tomando conta do objeto, mas não
mais que três, em um bunker de difícil acesso. Sua única chance de
entrar lá é explodindo uma escotilha que está escondida em algum lugar
da ilha. Você terá de parar o tempo para evitar que o guardião do
Receptáculo cumpra suas ordens de abri-lo no caso de problemas.
Lembre-se: é vital que ele não seja aberto.

Érico lutara contra a dor e ouvira a tudo atentamente. Ele percebera e
hesitação de Renata ao mencionar o fato de que não deveria abrir o
Receptáculo. Por que estavam escondendo uma informação importante como
aquela: Ele sabia o que aconteceria se a caixa fosse aberta, mas e se
não soubesse? Havia um detalhe a mais, algo que nem O Engenheiro nem
eles estavam dispostos a revelar facilmente. Talvez salvar o mundo ou
deixá-lo ser destruído não fossem as únicas opções no menu.

--- Deixa ver se eu entendi: --- disse ele. --- Eu tenho que achar uma
escotilha, que está escondida em uma ilha razoavelmente grande,
explodi-la para entrar em um bunker, parar o tempo e roubar uma caixa,
tudo isso sem abri-la em nenhum momento?

--- Trocando em miúdos, sim.

--- Fácil. Eu faria isso com as mãos atadas.

--- “Faria”: --- interveio Natália.

--- Futuro do pretérito, Natália. Quer dizer que eu não vou fazer.

--- Não sei se lembra, mas não tem escolha. Você irá sim fazer tudo o
que dissemos.

--- Enquanto eu dormia, Nat, eu sonhei. E fiz uma jornada pelo mundo dos
sonhos. Sabe como o lugar funciona? Não, não deve saber. Pode-se ver o
que desejar, do nosso universo e de outros, e eu quis ver um pouquinho
do que a Svenson vem fazendo. Durou milésimos de segundos, mas deu para
ver relances de criancinhas de apenas cinco anos trabalhando em 
fábricas de calcados da Svenson nas Filipinas por apenas alguns 
centavos de dólar por dia; políticos aqui no Brasil recebendo mesadas, 
algumas entregues por você, para deixar de aprovar leis que iriam de 
encontro com os interesses da companhia. Vi toda a sujeira que vocês 
despejam, os advogados que distorcem lei após lei para livrar sua 
cara, e toda a dor que é infligida a pessoas inocentes que ficam e m
seu caminho. Vocês representam o que há de pior no mundo, por isso não 
vou mover um dedo para ajudá-los. Podem me matar, se quiserem 

--- Pensamos que você diria algo assim. --- disse Natália, impassível.

--- Tragam-no.

Dois dos homens saíram da sala, e voltaram alguns segundos depois
carregando um desacordado José Betto.

--- Cortem a garganta dele. --- disse Natália.

O grito nem chegou a deixar a garganta de Érico; um dos homens deslizou
uma lâmina afiada pelo pescoço de Betto e, segundos depois de emitida a
ordem, ele jazia no chão em uma poça de sangue.

--- Você irá fazer o que mandamos, ou então cada pessoa que significa
algo para você terá o mesmo destino. --- sussurrou Natália, que estava
ao lado de Érico.

Érico não tinha palavras. Suas bravatas custaram a vida de seu primeiro
amigo em Franca, o homem que lhe dera apoio incondicional e lhe
apresentado um novo mundo. Uma nova lágrima brotou de seu olho direito,
e depois outra do esquerdo. Não houve, no entanto, mais tempo para
lástimas: a energia do prédio acabou no mesmo instante em que se pôde
ouvir um estrondo vindo do lado de fora da sala.

Érico tentou, em vão, se soltar para que pudesse fazê-los sofrer, mas as
amarras eram fortes demais. O silêncio sepulcral que se instalara na
sala foi rompido por vários “cliques” de armas sendo destravadas, e
rezou para que um tiro no escuro acabasse com tudo, com a imensa dor que
sentia pela responsabilidade de seus atos...

O novo estrondo foi quase ensurdecedor; a porta de aço que os separava
do corredor foi arrancada. Érico viu pela penumbra uma figura derrubar
  dois guardas telecineticamente com uma mão, enquanto parava as balas
  atiradas em sua direção com a outra. Com outro aceno, os homens
  portando armas da sala foram atirados em direção a parede, caindo em
  seguida desacordados. Apenas Natália e Renata continuavam de pé.

--- Não costumo bater em mulheres, --- disse Ian Abercrombie, entrando
na sala. --- mas vou abrir uma exceção para vocês.

Ian ergueu as duas com seu recém-adquirido poder, e arremessou-as uma
contra outra.

--- Nada quebrado, só alguns hematomas. Depois não diga que o
cavalheirismo morreu. --- disse ele, antes de olhar para o chão e ver
José morto. Seu semblante tornou-se sério. --- Acho que cheguei tarde
demais. Esse era o José? --- perguntou ele, enquanto soltava as amarras
de Érico.

--- Sim. Os desgraçados o mataram porque eu disse que não iria cooperar.

--- O que esse pessoal quer que você faça?

Érico contou tudo o que acontecera enquanto procuravam pela Kamen,
inclusive de sua viagem pelo mundo dos sonhos. Ian depois relatou sua
própria experiência, e como tinha obtido poderes de uma hora para outra.

--- Bem, acho que não temos outra opção a não ser ir até Jira. --- disse
Ian, após terem terminado seus relatórios. --- Lá decidiremos o que
fazer.

Érico ligou para Tio Baca, pedindo que ele buscasse o corpo de José e
providenciasse um funeral adequado.

--- Você ouviu o que eu disse, certo? Salvar o mundo não implica
mudá-lo; na verdade, significa deixar tudo como está. --- disse Érico,
montando na Kamen.

--- Não exatamente. --- disse Ian, seguindo Érico. --- Vamos a Jira. Eu
tenho um plano.
