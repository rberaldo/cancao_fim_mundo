\chapter{Flashback of Fools}

\data{Fazenda Ferraz, 2001}

--- Eu acredito no Brasil

Desmond Abercrombie não era um homem dado a sentimentalismos ou à pieguice,
como algum incauto depreenderia de suas primeiras palavras. Na realidade, nem
mesmo gostava de falar em público em um idioma que não fosse o seu, mas a
ocasião pedia um discurso. Estava diante de três investidores e suas
respectivas famílias, prestes a fechar o negócio de sua vida: seria o
fornecedor oficial de bebidas alcoólicas da próxima Copa do Mundo, o que lhe
propiciaria enormes lucros e visibilidade internacional. Por isso, venceu a
timidez e, com seu carregado sotaque escocês, prosseguiu:

--- Vejam bem, não quero parecer demagógico, eu realmente acredito nesse país
maravilhoso. Foi aqui que meu negócio se expandiu e diversificou, foi aqui que
conheci o amor de minha vida --- olhou enternecido para sua querida Laura, que
tossiu disfarçadamente. Sua saúde andava abalada, teriam de ir a um médico em
breve, logo que tudo aquilo acabasse. --- e foi aqui que constituí minha
família. Este país me ofereceu tanto, e agora é hora de contribuir. Para
atender a demanda vindoura, milhares de empregos serão criados em território
nacional tanto em caráter provisório quanto permanente, e parte de nosso lucro
será revertida para obras filantrópicas.

Todos os presentes aplaudiram com entusiasmo, exceto um: Ian bateu as mãos
apenas duas vezes, da forma mais irônica que pôde. Ele sabia que as doações
isentariam seu pai de pagar boa parte do imposto que deveria, e que todos os
empregados contratados seriam tão usados e explorados quanto os que já
trabalhavam para as indústrias Abercrombie. Seu pai, no entanto, se achava um
verdadeiro humanista, um homem de grande caráter que estava prestando um grande
favor ao país.

Ian não ignorava o fato, porém, de seu pai estar bem-intencionado. Fazia tudo
por amor a Laura, de uma maneira a não comprometer os negócios, mas fazia,
mesmo se iludindo com sua própria grandeza.

Fora Laura que pedira aquele favor em especial; ela mesma sempre dedicara muito
de seu tempo a causas nobres, fundando inclusive várias organizações em prol
dos necessitados. Agora, com a iminência desse novo e grandioso negócio, todos
os seus projetos seriam catapultados a um novo patamar. Seria esse o seu legado
para a Humanidade; estava fadada a partir em breve, sabia disso.

Era difícil esconder seus acessos de tosse, seguidos pelo sangramento que
irrompia de seu nariz e boca, de seu marido e filhos, mas eles apenas
visitavam-na na fazenda nos fins-de-semana. Ela insistira que precisava de ar
fresco e puro, um descanso das atribulações da cidade grande, mas a verdade era
que ela se mudara para a fazenda Ferraz, o lugar onde ela passou muitos de seus
verões na infância, para morrer.

Ela observou o aplauso cheio de animosidade de Ian, e sentiu pena por alguns
instantes. Seu filho mais novo estava profundamente infeliz por ter de partir
para a Escócia e ficar dois anos no Castelo Rowling, mas não havia outra
solução. Se pudesse, deixaria que fizesse o que bem entendesse sem nenhum
problema, mas as coisas não funcionavam assim em sua família. Leonardo, irmão
mais velho de Laura, jamais permitiria que ele escapasse de suas obrigações.

Em seguida olhou para Alden. Uma cópia do pai, em muitos sentidos. Era altivo,
competente, mas padecia do mesmo pecado que Desmond: a ambição. Ela temia que
seu filho mais velho, já versado nas artes da magia e profundamente envolvido
nos negócios da família, ultrapassasse os limites e se tornasse um homem
impiedoso e prepotente como seu avô Luís fôra.

Enquanto Laura pensava sobre ele, Alden fingia ouvir o discurso de seu pai.
Estava cansado de lidar com aquele homem que mal falava sua língua e beirava a
incompetência. Fora ele, Alden, quem havia costurado o acordo que seu pai
estava prestes a fechar, apenas utilizou-se do nome do pai para viabilizá-lo.
Mas ele, em sua arrogância e ignorância extremas, não pensou em incluir aquele
fato em seu discurso. Quando ele morresse, pensou Alden, toda a divisão de
bebidas alcoólicas que ele tanto amava seria desativada, vendida a preço de
banana para algum conglomeradozinho.

Alden, na verdade, já planejava toda sua vida após a morte de seus pais. Iria
manter a empresa como fachada enquanto se dedicaria à magia integralmente,
montando seu exército. Quanto a Ian, este provavelmente viveria drogado em
algum lugar, e mandaria a ele uma generosa para mantê-lo longe e distraído.

Sua mente flutuou para o segundo andar do casarão da fazenda. Sua mãe,
desconfiava, mantinha as joias de Circe em algum lugar de seu quarto,
escondidas de todos. Lembrava de ter visto o colar e a tiara muitas vezes, mas
só recentemente descobriu seus poderes: combinadas, as joias conferiam
invulnerabilidade e onisciência a quem as usasse. Ele, é claro, queria por as
mãos nessas preciosidades, mas sabia que teria de esperar até que o estado de
sua mãe piorasse. Ele parecia o único a desconfiar daquilo, mas ela estava
extremamente doente. Um feitiço hipocrático confirmara suas suspeitas; sua
morte era uma mera questão de tempo. Assim que pudesse, pegaria as joias para
si e se tornaria o mago mais poderoso do mundo.

Quando as palmas cessaram, Desmond foi até perto de sua esposa e beijou-lhe no
rosto. Quando morresse, dali a dois anos, suas últimas palavras seriam “Laura,
finalmente vou te rever.”

Ian levantou-se e saiu da sala. Achava que suas obrigações de manter as
aparências haviam sido diligentemente cumpridas e foi até seu quarto para ler
um pouco antes do jantar.

Laura rezou pedindo que ficasse tudo bem com ela até o fim da noite, sem
acessos de tosse ou nenhum outro acontecimento embaraçoso.

Alden foi até seu pai e lhe deu um abraço, torcendo em seu íntimo para que
aquele detestável homem saísse o quanto antes de seu caminho para a glória.
