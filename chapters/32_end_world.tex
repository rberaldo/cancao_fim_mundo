\chapter{\foreignlanguage{english}{The End of the World as We Know It}}

Ian Miguel Abercrombie e Érico Porto teleportaram-se para Jira graças à
Kamen, mas não puderam utilizar seus outros mecanismos, pois Alden
inibira o uso de quaisquer aparelhos eletrônicos com um poderoso
feitiço.

--- Não há nada que você possa fazer com esses seus novos poderes? ---
indagou um consternado Érico.

--- Não, infelizmente. Não sei o suficiente de linguagem mágica para
anular o feitiço

--- Não entendi.

--- Telecinesia, teleporte\ldots\,Essas são coisas que já vi sendo
feitas, portanto eu consigo reproduzir depois de apreender a informação
necessária. Magia, como eu descobri, tem a ver com manipular essa
informação de modo a alterar a realidade. Feitiços fazem parte dessa
categoria, mas são muito mais intrincados em sua concepção. Eu poderia
passar algumas horas aqui, desvendando cada encadeamento que me
permitiria anulá-lo, mas acho que não temos tempo para isso.

--- Cara, sinto saudade da época em que apenas eu falava coisas
incompreensíveis.

Ian deu uma pequena risada, enquanto Érico apenas sorriu. A morte de
José Betto ainda estava lhe doendo profundamente, mas sabia que teria de
seguir em frente em nome de seu amigo.

Embrenharam-se na floresta, deixando a Kamen na praia em que o feitiço
começava a agir. Érico pegara sua katana e um pacote de explosivos
plásticos que eram ativados por pressão. Ian seguia sem armas, atento
para algum truque a mais de seu irmão que poderia surpreendê-los.

\espaco

\data{Barretos, São Paulo}

O plano de Alden Abercrombie falhara. Seu irmão não havia sido
encontrado pelos agentes da \textsc{abin} liderados por Kátia Sinatra,
nem nem pelos magos que subornava da Ordem de Hermes. Apenas soubera,
depois de quase 24 horas de investigações, que Érico Porto
frequentemente era mencionado como amigo de um agente da \textsc{abin},
José Betto, e que ambos já tinham sido vistos entrando na residência de
um certo Tio Baca, supostamente um wuki.

Alden tentara obter a localização precisa de Tio Baca, mas isso se
mostrou impossível, pois um campo strangeliano protegia o lugar de
quaisquer investigações mágicas e atualmente outro campo, de distorção
de informação, tornava o lugar invisível a olho nu. Somente conseguiria
atingi-lo se detonasse uma bomba nas supostas proximidades de sua casa.

Estava irritado; não só perdera a chance de juntar as jóias de Circe
como também algumas peças de seu exército, os sacis que atacaram Ian e
Érico. Além disso, a Svenson continuava em seu encalço, embora não
pudessem feri-lo diretamente. Se o fizessem, ele não poderia ligar para
o Desmundo no horário programado, o que faria seu homem na ilha seguir a
diretiva de apertar o botão “Executar”, abrindo assim o Receptáculo e
destruindo o mundo.

Sim, o cerco estava se fechando, mas ele ainda tinha aquele truque na
manga.  Ninguém poderia aportar na ilha enquanto ela estivesse em seu
loop espaço-temporal, a não ser que já tivessem ido para lá. E, ele
garantira isso, ninguém que representasse ameaça alguma tinha.

Alden relaxou um pouco ao pensar naquilo e afrouxou sua gravata,
colocando um pouco do uísque que costumava ser fabricado pelas
Indústrias Abercrombie em um copo. Quando estava prestes a dar o
primeiro gole, seu sangue gelou: o telefone que nunca deveria tocar,
aquele conectado com a ilha de Jira, aquele que só tocaria em caso de
uma emergência extrema, tocou. Foi até ele, tremendo, e atendeu.

A voz, no entanto, não era de Desmundo e sim de seu irmão.

--- Olá, Alden. Seu garoto está desacordado. Fui gentil com ele. Já sei
de tudo, sobre o Receptáculo, sobre a Svenson, sobre seu exército e seu
planozinho de dominação mundial. Tudo isso começa a acabar aqui e agora.
Você tem 24 horas para liberar, sãs e salvas, todas as criaturas mágicas
que capturou. Depois eu e você discutiremos os rumos da empresa que
nosso pai nos deixou.

--- Seu\ldots

--- Não perca tempo com isso, Alden. Não vale a pena. Essa batalha está
perdida.  Nós continuaremos a guerra depois. Te vejo em outra vida,
irmão.

Click. A ligação havia sido interrompida. Alden sentou na poltrona, e
colocou a face nas mãos, quase chorando. Esperou ali por alguns minutos
pelos agentes da Svenson, que fatalmente descobririam aquilo e o
matariam.

Ninguém, no entanto, veio. E ninguém viria.

\espaco

\data{Jira}

Desmundo estava no chão. Recebera um duro golpe com o cabo da katana de
Érico, mas provavelmente acordaria bem. Ian sabia que seu irmão
provavelmente corrompera aquele homem com promessas de fortunas que
nunca seriam cumpridas, e que não deveria exagerar sem saber a extensão
de sua culpa.

Érico andou pela sala de computadores do bunker, e viu um cofre que era
conectado a um computador. O feitiço de Alden não incluía aquele lugar.
Sentiu um calafrio ao olhar para aquela caixa de metal, e desconfiou que
ali estava trancado o Receptáculo.

--- Ian, acho que\ldots

--- Sim. Afaste-se.

Ian concentrou-se para ler a informação ao redor do cofre. A senha, que
deveria ser colocada manualmente através de um botão giratório, estava
evidente para ele, e não havia quaisquer outras defesas. Foi até a porta
e girou o botão até que ela abrir. Dentro, só havia a caixa e dois
braços mecânicos ligados a um fio, que provavelmente a abririam caso o
computador mandasse. Ian fitou a caixa por vários minutos, até que
estendeu a mão em sua direção.

--- O que diabos você está fazendo? --- gritou Érico, extremamente
desconfortável com o gesto.

Ian não precisou dizer nada: com um abanar de mão, desconectou o fio e
afastou os braços de perto da caixa. Depois, trouxe-a telecineticamente
para uma mesa à sua frente.

--- Pensei em uma coisa. --- hesitou um pouco, mas continuou. --- Essa é
uma caixa de música, afinal de contas. Embora não tenha um mecanismo
interno que a faça tocar, ela é, em essência, uma caixa que toca uma
música. Música é informação acústica. E eu posso manipular informação.

--- O que está dizendo?

--- Posso fazer com que a música que sair daqui por um milionésimo de
segundo desfaça um objeto em particular. Depois disso, ela será jogada
no mar, onde ninguém poderá achá-la.

--- Que objeto você destruiria?

--- A pedra filosofal que está em poder da Svenson. Apenas isso. Sem
ela, eles serão mortais e deixarão de ser invencíveis.

--- Ok, e o que acontece depois disso?

--- A matéria irá virar informação, a ser espalhada pelo nosso mundo.
Sei que isso é possível, e que não causará danos.

--- Espero mesmo que tenha certeza.

--- Tem mais uma coisa. Todos no mundo sentirão esse efeito, de alguma
forma. A informação tocará alguma faculdade cognitiva das pessoas, de
modo que elas ficarão cientes de algo. É aí que você entra.

--- Não entendi.

--- Posso manipular informação, lembra? Da mesma forma que posso
controlar o que será destruído, dessa única vez, posso fazer com que as
pessoas “ouçam” o que eu quiser. Acho que poderemos enviar-lhes uma
mensagem, e não poderia pensar em ninguém melhor do que você para
concebê-la.

--- Não sei, cara. Isso pode não funcionar.

--- Por que está tão hesitante?

--- Não parece ser a coisa certa. Eu não quero ter essa
responsabilidade.

--- Eu também não queria, lembra?

--- Estamos falando de coisas diferentes, Ian.

--- Estamos mesmo. “Você pode salvar o mundo, mas será que consegue
mudá-lo?” Essa é a terceira opção, Érico. Eu sei que parece algo
totalitário, eu mesmo não aprovaria isso se não soubesse o que está em
jogo. Mas eu sinto que é o caminho correto. Só que não vou fazer isso
sem você. Esse é um trabalho para duas pessoas. Por que você acha tão
difícil acreditar?

--- E por que você acha tão fácil?

--- Nunca foi fácil! --- gritou Ian, exasperado.

Érico sabia que era sua vez de dar um salto de fé. Esperou um pouco
antes de dizer:

--- Vamos lá. Vamos mudar o mundo.

\espaco

\data{Londres. Sede da Svenson}

Os sete sócios estavam reunidos tomando brandy e discutindo os melhores
meios de acabar com Alden Abercrombie uma vez que seu fantoche, Érico
Porto, obtivesse o receptáculo. Mas sua pequena assembleia findou antes
do previsto, pois cada um de seus membros subitamente secou como uma
 ameixa e morreu. Os efeitos da Pedra Filosofal, que desaparecera
 momentos antes, cessaram, e os sete sócios da Svenson tiveram de pagar
 pela arrogância de terem tido uma vida mais longeva que o normal.

\espaco

\data{Em todo lugar}

Uma voz sussurrante instalou-se nas mentes de todas as pessoas em todas
as línguas. Não era alta ou imponente o suficiente para desviar a
atenção, mas estava lá. Mesmo quem não prestasse atenção iria lembrar-se
do que fora dito por ela:

\textit{Olá, cidadãos da Terra. As operações da Svenson de dominação
mundial estão prestes a ser fechadas. Este mundo será mais uma vez
nosso, assim como a responsabilidade será. Temos uma segunda chance de
fazer as coisas direito, de fazer o bem.}

\textit{Olhem ao seu redor e vejam o mundo que agora é seu, para fazer o que
quiserem dele, sem amarras ou medo. É hora de cada um de vocês começarem
a mudá-lo. Boa sorte a todos.}
